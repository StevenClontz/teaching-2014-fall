\documentclass[letterpaper, twoside, 12pt]{book}

\usepackage{amsmath,amsfonts, amsthm}
\usepackage{yfonts}
\usepackage{amsrefs}
\usepackage{fancyhdr}
\usepackage{graphicx}
\usepackage{float}
\usepackage[margin=1in]{geometry}
\usepackage{array}
\usepackage{esint}
\usepackage{harpoon}
\usepackage{stmaryrd}
\usepackage{multicol}
\usepackage{multirow}
\usepackage{pgf,tikz}
\usetikzlibrary{arrows}

\usepackage{makeidx}
\makeindex

\definecolor{AuburnOrange}{RGB}{221,85,12}
\definecolor{AuburnBlue}{RGB}{3,36,77}
\definecolor{AuburnSecondaryBlue}{RGB}{73,110,156}
\definecolor{AuburnSecondaryOrange}{RGB}{246,128,38}
\definecolor{AlabamaCrimson}{RGB}{163,38,65}
\definecolor{LSUpurple}{RGB}{70,29,124}
\definecolor{VanderbiltGold}{RGB}{207,181,59}

\usepackage[pdfpagelabels]{hyperref}
\hypersetup{colorlinks=true,linkcolor=AuburnSecondaryOrange}

% \usepackage[tracking]{microtype}
% \UseMicrotypeSet{all}
% \SetTracking[spacing = {35*,0*,0*}]{encoding = *}{7}
% \linespread{1.025}

\def\scaleint#1{\vcenter{\hbox{\scaleto[3ex]{\displaystyle\int}{#1}}}}
\def\bs{\!\!}

\renewcommand{\arraystretch}{1.5}

\pagestyle{fancy} \headheight 14.49998pt

\newcommand{\tstamp}{\today}
\renewcommand{\chaptermark}[1]{\markboth{#1}{}}
\renewcommand{\chaptermark}[1]{\markright{#1}}

\lhead[\fancyplain{}{Clontz \thepage}]         {\fancyplain{}{\scshape\nouppercase{\rightmark}}}

\chead[\fancyplain{}{}]
{\fancyplain{}{}}


\rhead[\fancyplain{}{Calculus II Lecture Notes}]       {\fancyplain{}{Clontz \thepage}}

\lfoot[\fancyplain{}{Auburn University}]                 {\fancyplain{\tstamp}{\tstamp}}

\cfoot[\fancyplain{\thepage}{}]         {\fancyplain{\thepage}{}}

\rfoot[\fancyplain{\tstamp} {\tstamp}]  {\fancyplain{}{Auburn University}}

\theoremstyle{definition}
\newtheorem{theorem}{Theorem}

% \theoremstyle{plain}
\newtheorem{proposition}[theorem]{Proposition}
\newtheorem{recall}[theorem]{Recall}

\theoremstyle{definition}
\newtheorem{definition}[theorem]{Definition}
\newtheorem{notation}[theorem]{Notation}
\newtheorem{goal}[theorem]{Goal}
\newtheorem{motivation}[theorem]{Motivation}
\newtheorem{remark}[theorem]{Remark}
\newtheorem{TrueFact}[theorem]{True Fact}
\newtheorem{FalseFact}[theorem]{False Fact}
\newtheorem{conjecture}[theorem]{Conjecture}
\newtheorem{conclusion}[theorem]{Conclusion}
\newtheorem{observation}[theorem]{Observation}
\newtheorem{problem}[theorem]{Problem}
\newtheorem{question}[theorem]{Question}
\newtheorem{example}[theorem]{Example}
\newtheorem{note}[theorem]{Note}
\newtheorem{convention}[theorem]{Convention}
\newtheorem{eqn}[theorem]{Equation}
\newtheorem{strategy}[theorem]{Strategy}
\newtheorem{properties}[theorem]{Properties}
\newtheorem{corollary}[theorem]{Corollary}

\newcommand{\HRule}{\rule{\linewidth}{0.5mm}}
\newcommand{\harpvec}[1]{\overrightharp{\ensuremath{\mathbf{#1}}}}
\newcommand*{\threevec}[3]{\ensuremath{\left\langle #1, #2, #3 \right\rangle}}
\newcommand*{\twovec}[2]{\ensuremath{\left\langle #1, #2 \right\rangle}}
\newcommand*{\unitvec}[1]{\ensuremath{\mathbf{\widehat{#1}}}}
\newcommand{\veci}{\ensuremath{\mathbf{\widehat{i}}}}
\newcommand{\vecj}{\ensuremath{\mathbf{\widehat{j}}}}
\newcommand{\veck}{\ensuremath{\mathbf{\widehat{k}}}}
\newcommand{\ds}{\ensuremath{\displaystyle}}


\newcommand{\contrasymb}{\mathrel{\raisebox{.1em}{\reflectbox{\rotatebox[origin=c]{220}{$\lightning$}}}}}

\newenvironment{answer}{\paragraph{Answer.}}{\hfill$\blacklozenge$}
\newenvironment{contraproof}{\paragraph{Proof.}}{\hfill$\contrasymb$}

\begin{document}

\setcounter{chapter}{6}

\chapter{Techniques of Integration}

\section{Integration by Parts}

\begin{problem}
 Prove that
 $\ds \int f(x)g^\prime(x) \, dx = f(x)g(x) - \int g(x)f^\prime(x) \, dx.$
 \emph{Hint}  Use the product rule and work backwards.
\end{problem}

\vfill

\begin{theorem}[Integration by Parts\index{Integration by Parts}]
 Given two continuous, differentiable functions $f(x)$ and $g(x)$,
 $$\int f(x)g^\prime(x) \, dx = f(x)g(x) - \int g(x)f^\prime(x) \, dx$$
 If $u = f(x)$ and $v = g(x)$, then we can write this as
 $$\int u \, dv = uv - \int v \, du$$
\end{theorem}

\begin{problem}
 Evaluate $\ds \int x\sin(x) \, dx$.
\end{problem}

\vfill

\newpage

\begin{problem}
 Evaluate $\ds \int \ln(x) \, dx$.
\end{problem}

\vfill

\begin{problem}
 Evaluate $\ds \int t^2e^t \, dt$.
\end{problem}

\vfill

\newpage

\begin{problem}
 Evaluate $\ds \int_0^1 \arctan(x) \, dx$.
\end{problem}

\vfill

\begin{problem}
 Evaluate $\ds \int e^x \sin(x) \, dx$.
\end{problem}

\vfill

\noindent Suggested Homework: Section $7.1$ numbers $1 - 4,$ $7,$ $10 - 12,$ $21,$ $24,$ $29,$ $30,$ $31$

\newpage

\section{Trigonometric Integrals}

\subsection{Products of Powers of Sine and Cosine}

\begin{strategy}

  There are three types of integrals of the form $\int \sin^m(x) \cos^n(x) dx$:

  \begin{enumerate}
  \renewcommand{\theenumi}{\Roman{enumi}}
  \item \textbf{The power on $\sin(x)$ is odd.}

    Apply $\sin^{2n+1}(x)=(\sin^2(x))^n\sin(x)=(1-\cos^2(x))^n\sin(x)$
    and use the substitution $u=\cos(x)$.

  \item \textbf{The power on $\cos(x)$ is odd.}

    Apply $\cos^{2n+1}(x)=(\cos^2(x))^n\cos(x)=(1-\sin^2(x))^n\cos(x)$
    and use the substitution $u=\sin(x)$.

  \item \textbf{Both powers are even.}

    Apply both $\cos^{2n}(x)=\left(\frac{1+\cos(2x)}{2}\right)^n$ and
    $\sin^{2n}(x)=\left(\frac{1-\cos(2x)}{2}\right)^n$ to reduce the exponents
    in the integral.
  \end{enumerate}

\end{strategy}

\begin{problem}
 Evaluate $\ds \int \cos^3(x) \, dx$.
\end{problem}

\vfill

\begin{problem}
 Evaluate $\ds \int \sin^5(x)\cos^2(x) \, dx$.
\end{problem}

\vfill

\newpage

\begin{problem}
 Evaluate $\ds \int_0^{\frac{\pi}{4}} \sin^2(x) \, dx$.
\end{problem}

\vfill

\begin{problem}
 Evaluate $\ds \int \sin^4(x) \, dx$.
\end{problem}

\vfill

\subsection{Products of Powers of Tangent and Secant}

\begin{strategy}
 To evaluate an integral of the form $\int \tan^m(x)\sec^n(x) \, dx$:
 \begin{itemize}
  \item If $n$ is even,
  \begin{itemize}
    \item Save a factor of $\sec^2(x)$ and use $\sec^2(x) = 1+ \tan^2(x)$
          on the rest.
    \item Use the $u$ substitution $u = \tan(x)$.
  \end{itemize}
  \item If $m$ is odd,
  \begin{itemize}
    \item Save a factor of $\sec(x)\tan(x)$ and use $\tan^2(x) = \sec^2(x) -1$
          on the rest.
    \item Use the $u$ substitution $u = \sec(x)$.
  \end{itemize}
 \end{itemize}
\end{strategy}

\begin{problem}
 Evaluate $\ds \int \tan^6(x)\sec^4(x) \, dx$.
\end{problem}

\vfill

\newpage

\begin{problem}
 Evaluate $\ds \int \tan^5(\theta)\sec^7(\theta) \, d\theta$.
\end{problem}

\vfill

\begin{recall}
 $\ds \int \tan(x) \, dx = \ln\left|\sec(x)\right| + c$ and $\ds \int \sec(x) \, dx = \ln\left|\sec(x) + \tan(x)\right|+c$
\end{recall}

\begin{problem}
 Evaluate $\ds \int \tan^3(x) \, dx$.
\end{problem}

\vfill

\newpage

\begin{problem}
 Use Integration by Parts to evaluate $\ds \int \sec^3(x) \, dx$.
\end{problem}

\vfill

\begin{recall}
$$\begin{array}{rl}
    \sin(A)\cos(B) &= \frac{1}{2} \left[\sin\left(A - B\right) + \sin\left(A + B\right)\right] \\
    \sin(A)\sin(B) &= \frac{1}{2} \left[\cos\left(A - B\right) - \cos\left(A + B\right)\right] \\
    \cos(A)\cos(B) &= \frac{1}{2} \left[\cos\left(A - B\right) + \cos\left(A + B\right)\right].
  \end{array}$$
\end{recall}

\begin{problem}
 Evaluate $\ds \int \sin(4x)\cos(5x) \, dx$.
\end{problem}

\vfill

\noindent Suggested Homework: Section $7.2$ numbers $1,$ $3,$ $5 - 7,$ $10,$ $11,$ $15,$ $21,$ $23,$ $25,$ $27,$ $29,$ $38$

\newpage

\section{Trigonometric Substitution}

\begin{strategy}

With square roots and other troublesome factors, it sometimes helps to
substitute trigonometric functions in order to use their identities for
cancellation.

$$\begin{array}{|c|c|c|c|}
   \hline
   \mbox{Expression} & \mbox{Substitution} &
   \mbox{Differential} & \mbox{Fact to Use} \\
   \hline

     a^2-x^2 &
     x=a\sin(\theta) \Rightarrow x^2 = a^2\sin^2(\theta) &
     dx=a\cos(\theta)d\theta &
     1-\sin^2(\theta) = \cos^2(\theta) \\

     a^2+x^2 &
     x=a\tan(\theta) \Rightarrow x^2 = a^2\tan^2(\theta) &
     dx=a\sec^2(\theta)d\theta &
     1+\tan^2(\theta) = \sec^2(\theta) \\

     x^2-a^2 &
     x=a\sec(\theta) \Rightarrow x^2 = a^2\sec(\theta) &
     dx=a\sec(\theta)\tan(\theta)d\theta &
     \sec^2(\theta)-1 = \tan^2(\theta) \\
   \hline
  \end{array}$$

\end{strategy}

\begin{problem}
 Prove $\ds \int \frac{1}{1+x^2}\, dx = \arctan(x)+C$.
\end{problem}

\vfill

\begin{problem}
 Evaluate $\ds \int \frac{\sqrt{9-x^2}}{x^2} \, dx$.
\end{problem}

\vfill

\newpage

\begin{problem}
 Evaluate $\ds \int \frac{2x}{x^2+1}\, dx$.
\end{problem}

\vfill

\begin{problem}
 Evaluate $\ds \int \frac{1}{x^2\sqrt{x^2+4}} \, dx$.
\end{problem}

\vfill

\newpage

\begin{problem}
 Evaluate $\ds \int \frac{x}{\sqrt{x^2 + 4}} \, dx$.
\end{problem}

\vfill

\begin{problem}
 Evaluate $\ds \int_0^{\frac{3\sqrt{3}}{2}} \frac{x^3}{\left(4x^2+9\right)^{3/2}} \, dx$.
\end{problem}

\vfill

\newpage

\begin{problem}
 Evaluate $\ds \int \frac{dx}{\sqrt{x^2-a^2}}$ for $a >0$.
\end{problem}

\vfill

\begin{problem}
 Evaluate $\ds \int \frac{1}{\sqrt{3-2x-x^2}} \, dx$.  \emph{Hint} Complete the square.
\end{problem}

\vfill

\noindent Suggested Homework: Section $7.3$ numbers $2,$ $4,$ $5,$ $7,$ $9,$ $10,$ $11,$ $16,$ $22$

% \newpage

% \section{Partial Fraction Decomposition}

% \begin{goal}
%  To be able to integrate {\sc any} rational function!
% \end{goal}

% \begin{theorem}[Fundamental Theorem of Algebra\index{Fundamental Theorem of Algebra}]
%  Every polynomial in $\mathbb{R}$ is factorable into linear terms and irreducible quadratics.
% \end{theorem}

% \begin{strategy}
%  If the degree in the numerator is greater than or equal to the degree in the denominator, use long division.
% \end{strategy}

% \begin{problem}
%  Evaluate $\ds \int \frac{x^3 + x}{x-1} \, dx$.
% \end{problem}

% \vfill

% \begin{motivation}
%  If we were asked to add fractions, we would do the following: $$\frac{2}{x} + \frac{1}{x+1} = \frac{2(x+1)}{x(x+1)} + \frac{1(x)}{(x+1)(x)} = \frac{2x+2}{x(x+1)} + \frac{x}{x(x+1)} = \frac{2x+2+x}{x(x+1)} = \frac{3x+2}{x^2+x}.$$  What we would like to do is to figure out how to do this same procedure in reverse.
% \end{motivation}

% \vspace{1in}

% \newpage

% \begin{strategy}
%  If the denominator is the product of distinct linear factors, split the linear factors up and solve the system of equations.
% \end{strategy}

% \begin{problem}
%  Evaluate $\ds \int \frac{x^2+2x-1}{2x^3+3x^2-2x} \, dx$.
% \end{problem}

% \vfill

% \newpage

% \begin{strategy}
%  If the denominator is a product of linear factors, some of which are repeated roots, split the factors adding one to the degree of the repeated root until it is of the desired power.
% \end{strategy}

% \begin{problem}
%  Evaluate $\ds \int \frac{x^4-2x^2+4x+1}{x^3 - x^2 -x+1} = \int x + 1 + \frac{4x}{x^3 - x^2 -x+1} \, dx$. \emph{Notice} that long division on the first integral produces the second.
% \end{problem}

% \vfill

% \newpage

% \begin{strategy}
%  If the denominator has an irreducible quadratic, make the numerator of degree one less.
% \end{strategy}

% \begin{problem}
%  Evaluate $\ds \int \frac{3x^2-x+4}{x^3 + 4x} \, dx$.
% \end{problem}

% \vfill

% \newpage

% \begin{strategy}
%  If the denominator contains repeated irreducible, treat the repeated roots as before with the denominator of degree one less than the denominator.
% \end{strategy}

% \begin{problem}
%  Evaluate $\ds \int \frac{1-x+2x^2-x^3}{x(x^2+1)^2} \, dx$
% \end{problem}

% \vfill

% \noindent Suggested Homework: Section $7.4$ numbers $2,$ $3,$ $12,$ $15,$ $16,$ $17,$ $18,$ $19,$ $26,$ $27$

% \newpage

% \setcounter{section}{7}
% \section{Improper Integrals}

% \subsection{Infinite Intervals}

% \begin{definition}[Integral Over an Infinite Interval\index{Integral Over Infinite Interval}]
% The following will help us evaluate integrals without bound in at least one direction:
%  \begin{itemize}
%   \item If $\int_a^t f(x) \, dx$ exists for every $t \geq a$, then $\int_a^\infty f(x) \, dx = \lim_{t \rightarrow \infty} \int_a^t f(x) \, dx = L$ provided $L \in \mathbb{R}$ (that is, $L$ cannot be $\pm \infty$).
%   \item If $\int_t^b f(x) \, dx$ exists for every $t < b$, then $\int_{-\infty}^b f(x) \, dx = \lim_{t \rightarrow -\infty} \int_t^b f(x) \, dx = L$ provided $L \in \mathbb{R}$ (that is, $L$ cannot be $\pm \infty$).
%   \item If both $\int_a^\infty f(x) \, dx$ and $\int_{-\infty}^b f(x) \, dx$ exist, then $\int_{-\infty}^{\infty} f(x) \, dx= \int_a^\infty f(x) \, dx + \int_{-\infty}^b f(x) \, dx$
%  \end{itemize}
%  If $L \in \mathbb{R}$ and not $\pm \infty$, then the integral is said to converge.  If the limit does not exist or $L = \pm \infty$, then the integral is said to diverge.
% \end{definition}

% \begin{problem}
%  Find the area under the curve $\ds y = \frac{1}{x^2}$ from $x=1$ and onward.
% \end{problem}

% \vfill

% \begin{problem}
%  Determine whether $\ds \int_1^\infty \frac{1}{x} \, dx$ is convergent or divergent.
% \end{problem}

% \vfill

% \newpage

% \begin{problem}
%  Determine whether $\ds \int_{-\infty}^0 xe^x \, dx$ is convergent or divergent.
% \end{problem}

% \vfill

% \begin{problem}
%  Determine whether $\ds \int_{-\infty}^\infty \frac{1}{1+x^2} \, dx$ is convergent or divergent.
% \end{problem}

% \vfill

% \begin{problem}
%  For what values of $p$ is the integral $\ds \int_1^\infty \frac{1}{x^p} \, dx$ convergent?
% \end{problem}

% \vfill

% \newpage

% \subsection{Discontinuous Integrals}

% \begin{definition}[Integral Over Discontinuous Interval\index{Integral Over Discontinuous Interval}]
% The following will help us evaluate integrals that are not continuous over a given interval:
%  \begin{itemize}
%   \item If $f$ is continuous on $\left[a,b\right)$ and discontinuous at $b$, then $\int_a^b f(x) \, dx = \lim_{t \rightarrow b^-} \int_a^t f(x) \, dx = L$ provided $L \in \mathbb{R}$ (that is, $L$ cannot be $\pm \infty$).
%   \item If $f$ is continuous on $\left(a,b\right]$ and discontinuous at $a$, then $\int_a^b f(x) \, dx = \lim_{t \rightarrow a^+} \int_t^b f(x) \, dx = L$ provided $L \in \mathbb{R}$ (that is, $L$ cannot be $\pm \infty$).
%   \item If $f$ is discontinuous at $c$ where $a < c < b$ and both $\int_a^c f(x) \, dx$ and $\int_c^b f(x) \, dx$ are convergent, then $\int_{a}^{b} f(x) \, dx = \int_a^c f(x) \, dx + \int_{c}^b f(x) \, dx$
%  \end{itemize}
%  If $L \in \mathbb{R}$ and not $\pm \infty$, then the integral is said to converge.  If the limit does not exist or $L = \pm \infty$, then the integral is said to diverge.
% \end{definition}

% \begin{problem}
%  Determine whether $\ds \int_2^5 \frac{1}{\sqrt{x-2}} \, dx$ converges or diverges.
% \end{problem}

% \vfill

% \begin{problem}
%  Determine whether $\ds \int_0^{\frac{\pi}{2}} \sec(x) \, dx$ converges or diverges.
% \end{problem}

% \vfill

% \newpage

% \begin{problem}
%  Determine whether $\ds \int_0^1 \ln(x) \, dx$ converges or diverges.
% \end{problem}

% \vfill

% \begin{problem}
%  Determine whether $\ds \int_0^3 \frac{1}{x-1} \, dx$ converges or diverges.
% \end{problem}

% \vfill

% \newpage

% \subsection{Comparison Test for Integrals}

% \begin{theorem}[Comparison Test for Integrals\index{Integral Comparison Test}]
%  Suppose that $f$ and $g$ are continuous functions with $0 \leq g(x) \leq f(x)$ for $x \geq a$.
%  \begin{itemize}
%   \item If $\int_a^\infty f(x) \, dx$ is convergent, then $\int_a^\infty g(x) \, dx$ is also convergent.
%   \item If $\int_a^\infty g(x) \, dx$ is divergent, then $\int_a^\infty f(x) \, dx$ is also divergent.
%  \end{itemize}
% \end{theorem}

% \begin{problem}
%  Show that $\ds \int_0^\infty e^{-x^2} \, dx$ is convergent.  \emph{Note} $\ds \int e^{-x^2} \, dx$ has no closed form integral.
% \end{problem}

% \vfill

% \begin{problem}
%  Determine whether $\ds \int_1^\infty \frac{1+e^{-x}}{x} \, dx$ converges or diverges.
% \end{problem}

% \vfill

% \noindent Suggested Homework: Section $7.8$ numbers $1,$ $7,$ $9,$ $13,$ $14 - 16,$ $18,$ $25,$ $27 - 33,$ $35,$ $49 - 52,$ $54,$ $55,$ $57$

\end{document}