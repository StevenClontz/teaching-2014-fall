\documentclass[letterpaper, twoside, 12pt]{book}

\usepackage{amsmath,amsfonts, amsthm}
\usepackage{yfonts}
\usepackage{amsrefs}
\usepackage{fancyhdr}
\usepackage{graphicx}
\usepackage{float}
\usepackage[margin=1in]{geometry}
\usepackage{array}
\usepackage{esint}
\usepackage{harpoon}
\usepackage{stmaryrd}
\usepackage{multicol}
\usepackage{multirow}
\usepackage{pgf,tikz}
\usetikzlibrary{arrows}

\usepackage{makeidx}
\makeindex

\definecolor{AuburnOrange}{RGB}{221,85,12}
\definecolor{AuburnBlue}{RGB}{3,36,77}
\definecolor{AuburnSecondaryBlue}{RGB}{73,110,156}
\definecolor{AuburnSecondaryOrange}{RGB}{246,128,38}
\definecolor{AlabamaCrimson}{RGB}{163,38,65}
\definecolor{LSUpurple}{RGB}{70,29,124}
\definecolor{VanderbiltGold}{RGB}{207,181,59}

\usepackage[pdfpagelabels]{hyperref}
\hypersetup{colorlinks=true,linkcolor=AuburnSecondaryOrange}

% \usepackage[tracking]{microtype}
% \UseMicrotypeSet{all}
% \SetTracking[spacing = {35*,0*,0*}]{encoding = *}{7}
% \linespread{1.025}

\def\scaleint#1{\vcenter{\hbox{\scaleto[3ex]{\displaystyle\int}{#1}}}}
\def\bs{\!\!}

\renewcommand{\arraystretch}{1.5}

\pagestyle{fancy} \headheight 14.49998pt

\newcommand{\tstamp}{\today}
\renewcommand{\chaptermark}[1]{\markboth{#1}{}}
\renewcommand{\chaptermark}[1]{\markright{#1}}

\lhead[\fancyplain{}{Clontz \thepage}]         {\fancyplain{}{\scshape\nouppercase{\rightmark}}}

\chead[\fancyplain{}{}]
{\fancyplain{}{}}


\rhead[\fancyplain{}{Calculus II Lecture Notes}]       {\fancyplain{}{Clontz \thepage}}

\lfoot[\fancyplain{}{Auburn University}]                 {\fancyplain{\tstamp}{\tstamp}}

\cfoot[\fancyplain{\thepage}{}]         {\fancyplain{\thepage}{}}

\rfoot[\fancyplain{\tstamp} {\tstamp}]  {\fancyplain{}{Auburn University}}

\theoremstyle{definition}
\newtheorem{theorem}{Theorem}

% \theoremstyle{plain}
\newtheorem{proposition}[theorem]{Proposition}
\newtheorem{recall}[theorem]{Recall}

\theoremstyle{definition}
\newtheorem{definition}[theorem]{Definition}
\newtheorem{notation}[theorem]{Notation}
\newtheorem{goal}[theorem]{Goal}
\newtheorem{motivation}[theorem]{Motivation}
\newtheorem{remark}[theorem]{Remark}
\newtheorem{TrueFact}[theorem]{True Fact}
\newtheorem{FalseFact}[theorem]{False Fact}
\newtheorem{conjecture}[theorem]{Conjecture}
\newtheorem{conclusion}[theorem]{Conclusion}
\newtheorem{observation}[theorem]{Observation}
\newtheorem{problem}[theorem]{Problem}
\newtheorem{question}[theorem]{Question}
\newtheorem{example}[theorem]{Example}
\newtheorem{note}[theorem]{Note}
\newtheorem{convention}[theorem]{Convention}
\newtheorem{eqn}[theorem]{Equation}
\newtheorem{strategy}[theorem]{Strategy}
\newtheorem{properties}[theorem]{Properties}
\newtheorem{corollary}[theorem]{Corollary}

\newcommand{\HRule}{\rule{\linewidth}{0.5mm}}
\newcommand{\harpvec}[1]{\overrightharp{\ensuremath{\mathbf{#1}}}}
\newcommand*{\threevec}[3]{\ensuremath{\left\langle #1, #2, #3 \right\rangle}}
\newcommand*{\twovec}[2]{\ensuremath{\left\langle #1, #2 \right\rangle}}
\newcommand*{\unitvec}[1]{\ensuremath{\mathbf{\widehat{#1}}}}
\newcommand{\veci}{\ensuremath{\mathbf{\widehat{i}}}}
\newcommand{\vecj}{\ensuremath{\mathbf{\widehat{j}}}}
\newcommand{\veck}{\ensuremath{\mathbf{\widehat{k}}}}
\newcommand{\ds}{\ensuremath{\displaystyle}}


\newcommand{\contrasymb}{\mathrel{\raisebox{.1em}{\reflectbox{\rotatebox[origin=c]{220}{$\lightning$}}}}}

\newenvironment{answer}{\paragraph{Answer.}}{\hfill$\blacklozenge$}
\newenvironment{contraproof}{\paragraph{Proof.}}{\hfill$\contrasymb$}

\begin{document}


\setcounter{chapter}{7}

\chapter{Further Applications of Integrals}

\section{Arc Length}

\begin{theorem}[Arc Length\index{Arc Length}]
  If $\frac{df}{dx}$ is continuous on $\left[a,b\right]$, then the length of the
  curve $y = f(x)$ where $a \leq x \leq b$ is
  \[
    L
      =
    \lim_{n\to\infty}\sum_{i=1}^n \sqrt{(\Delta x)^2+(\Delta f)^2}
      =
    \lim_{n\to\infty}\sum_{i=1}^n
    \sqrt{1+\left(\frac{\Delta f}{\Delta x}\right)^2}\,\Delta x
      =
    \int_a^b \sqrt{1+\left(\frac{df}{dx}\right)^2} \, dx
  \]
\end{theorem}

\begin{problem}
  Prove that the circumference of a circle with radius $r$ is
  $C=2\pi r$.
\end{problem}

\vfill

\begin{problem}
  Find the length of the arc on the curve $y^2 = x^3$ between
  the points $(1,1)$ and $(4,8)$.
\end{problem}

\vfill

\newpage

\begin{problem}
 Find the length of the arc of the parabola $y^2=x$ from $(0,0)$ to $(1,1).$
\end{problem}

\vfill

\begin{theorem}[Arc Length Function\index{Arc Length Function}]
  If $\frac{df}{dx}$ is continuous, then the \textbf{arc length function} with
  initial point $(a,f(a))$ for the curve $y = f(x)$ is
  \[
    s(x)
      =
    \int_a^x \sqrt{1+\left(f'(t)\right)^2} \, dt
  \]
\end{theorem}

\begin{problem}
 Find the arc length function for the curve $\ds y = x^2-\frac{1}{8}\ln(x)$
 taking $(1,1)$ as the initial point.
\end{problem}

\vfill

\noindent Suggested Problems Section $8.1$ numbers $1,$ $2,$ $5,$ $7,$ $8,$ $10,$ $11,$ $13,$ $14,$ $19,$ $20,$ $35$

\newpage

\section{Area of a Surface of Revolution}

\begin{theorem}[Surface Area\index{Surface Area}]
  Let $f$ be a positive function with continuous derivative.
  Then the area of the surface obtained by rotating the
  curve $y = f(x)$ from $a \leq x \leq b$ about the $x$-axis is
  \[
    SA
      =
    \int_a^b 2\pi f(x) \sqrt{1+\left[f^\prime\left(x\right)\right]^2} \, dx
  \]
\end{theorem}

\begin{problem}
  Prove that the surface area of a sphere with radius $r$ is given by
  $SA=4\pi r^2$
\end{problem}

\vfill

\begin{problem}
  Find the area of the surface generated by rotating the arc of the parabola
  $y = x^2$ from $(1,1)$ to $(2,4)$ about the $y$-axis.
\end{problem}

\vfill

\newpage

\begin{problem}
 Find the area of the surface generated by rotating $y=e^x$
 from $0 \leq x \leq 1$ about the $x$-axis.
\end{problem}

\vfill

\begin{problem}[Gabriel's Horn\index{Gabriel's Horn}]
  Show that the solid obtained by rotating the region
  bounded by the curve $y=\frac{1}{x}$ and lines $y=0$, $x=1$ about the
  $x$-axis has infinite volume and finite surface area.
\end{problem}

\vfill

\noindent Suggested Homework: Section $8.2$ numbers $5,$ $6,$ $7,$ $9,$ $13,$ $14,$ $16$

\end{document}