\documentclass[11pt]{article}

\pdfpagewidth 8.5in
\pdfpageheight 11in

\setlength\topmargin{0in}
\setlength\headheight{0in}
\setlength\headsep{0.4in}
\setlength\textheight{8in}
\setlength\textwidth{6in}
\setlength\oddsidemargin{0in}
\setlength\evensidemargin{0in}
\setlength\parindent{0.25in}
\setlength\parskip{0.1in}

% \usepackage{amssymb}
% \usepackage{amsfonts}
% \usepackage{amsmath}
% \usepackage{mathtools}
% \usepackage{amsthm}

\usepackage{fancyhdr}



\pagestyle{fancy}
% \renewcommand{\headrulewidth}{0.5pt}
% \renewcommand{\footrulewidth}{0pt}
% \lfoot{\small Updated on \today}
% \chead{\small Calculus II -- Mr. Clontz -- Fall 2014}
% \rfoot{\thepage}
% \cfoot{}



\begin{document}

\title{
  Calculus II -- Mr. Clontz -- Fall 2014 \\
  Participation / Quizzes / Exams
}

\maketitle

Participation and quizzes are a large portion (500 of 800 points) of your
grade. Here's a rubric for how those points will be assigned.

\section{Participation}

Participation will be split equally into three categories, each worth 100
points: \textbf{Attendance}, \textbf{Individual Presentations}, and
\textbf{Class Participation}.

\subsection{Attendance}

Each student is allowed one unexcused absence without penalizing their
Attendance grade. Two or more unexcused absences will result in the grade given
on the following table. Five unexcused absences will automatically result in an
FA for the course.

\begin{center}
\begin{tabular}{rl}
0-1 Unexcused & 100 points \\\hline
2 Unexcused & 90 points \\\hline
3 Unexcused & 60 points \\\hline
4 Unexcused & 30 points \\\hline
5+ Unexcused & 0 points \& FA
\end{tabular}
\end{center}

\subsection{Individual Presentations}

The instructor will assign reading and problems from the lecture notes for the
class to prepare for presenting at the board. During later class periods,
the instrutor will semi-randomly choose students to present previously
assigned problems, with a preference for students who have less credit for
presentations than others.

A student who successfully gives a prepared presentation earns 2 points.
If a student is not prepared and declines to present, they will not gain
or lose points.
If a student is not prepared and wastes the class's time by trying to fake it,
they will lose 1 point.

\textbf{Presentations don't have to be perfect} - the instructor is looking for
effort and preparation.
Students are encouraged to use their textbook, search the internet, work
with each other, go to office hours, or use any other source to solve the
assigned problems for presentation.

At the end of the semester, these points will be normalized so that the
student with the most points gets 100 points for individual presentations,
and other students' scores will be scaled appropriately.

\subsection{Class Participation}

The class has a shared Participation grade of 100 points.
If no one in the class is prepared to present an assigned problem, the
instructor may reduce this grade for everyone by 10 points per problem.


\section{Quizzes}

The class will be given a quick quiz at the start of the first day of class
each week. Each quiz will be worth around 20 points. With about 14 quizzes,
there will be approximately 280 points possible, but this score is capped at
200 points towards each student's final overall grade.

Quizzes will involve problems similar to those presented from the lecture
notes, or problems which have been assigned to be presented.


\newpage
\section{Exams}

\subsection{Midterm}

A midterm exam will be given on Monday, October 6th. This exam will be worth
100 points, and will test over material covered in lecture through roughly
Friday, September 26.

A practice midterm will be given out on Monday, September 29, which will
outline all possible questions which may be asked on the midterm, as well
as how partial credit will be assigned for each question. A review
will be held on Friday, October 3rd, where students can ask questions, and
other students can present solutions. Students may submit their full solutions
to the practice midterm before taking the actual midterm
on Monday, October 6th for up to 10 bonus points.

\subsection{Final}

A final exam will be given on Thursday, December 11th at 7pm. It will be worth
200 points toward your overall grade. This document
will be updated with details as that date approaches.


\end{document}