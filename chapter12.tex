\documentclass[letterpaper, twoside, 12pt]{book}

\usepackage{amsmath,amsfonts, amsthm}
\usepackage{yfonts}
\usepackage{amsrefs}
\usepackage{fancyhdr}
\usepackage{graphicx}
\usepackage{float}
\usepackage[margin=1in]{geometry}
\usepackage{array}
\usepackage{esint}
\usepackage{harpoon}
\usepackage{stmaryrd}
\usepackage{multicol}
\usepackage{multirow}
\usepackage{pgf,tikz}
\usetikzlibrary{arrows}

\usepackage{makeidx}
\makeindex

\definecolor{AuburnOrange}{RGB}{221,85,12}
\definecolor{AuburnBlue}{RGB}{3,36,77}
\definecolor{AuburnSecondaryBlue}{RGB}{73,110,156}
\definecolor{AuburnSecondaryOrange}{RGB}{246,128,38}
\definecolor{AlabamaCrimson}{RGB}{163,38,65}
\definecolor{LSUpurple}{RGB}{70,29,124}
\definecolor{VanderbiltGold}{RGB}{207,181,59}

\usepackage[pdfpagelabels]{hyperref}
\hypersetup{colorlinks=true,linkcolor=AuburnSecondaryOrange}

% \usepackage[tracking]{microtype}
% \UseMicrotypeSet{all}
% \SetTracking[spacing = {35*,0*,0*}]{encoding = *}{7}
% \linespread{1.025}

\def\scaleint#1{\vcenter{\hbox{\scaleto[3ex]{\displaystyle\int}{#1}}}}
\def\bs{\!\!}

\renewcommand{\arraystretch}{1.5}

\newcommand{\<}{\langle}
\renewcommand{\>}{\rangle}

\pagestyle{fancy} \headheight 14.49998pt

\newcommand{\tstamp}{\today}
\renewcommand{\chaptermark}[1]{\markboth{#1}{}}
\renewcommand{\chaptermark}[1]{\markright{#1}}

\lhead[\fancyplain{}{Clontz \thepage}]         {\fancyplain{}{\scshape\nouppercase{\rightmark}}}

\chead[\fancyplain{}{}]
{\fancyplain{}{}}


\rhead[\fancyplain{}{Calculus II Lecture Notes}]       {\fancyplain{}{Clontz \thepage}}

\lfoot[\fancyplain{}{Auburn University}]                 {\fancyplain{\tstamp}{\tstamp}}

\cfoot[\fancyplain{\thepage}{}]         {\fancyplain{\thepage}{}}

\rfoot[\fancyplain{\tstamp} {\tstamp}]  {\fancyplain{}{Auburn University}}

\theoremstyle{definition}
\newtheorem{theorem}{Theorem}

% \theoremstyle{plain}
\newtheorem{proposition}[theorem]{Proposition}
\newtheorem{recall}[theorem]{Recall}

\theoremstyle{definition}
\newtheorem{definition}[theorem]{Definition}
\newtheorem{notation}[theorem]{Notation}
\newtheorem{goal}[theorem]{Goal}
\newtheorem{motivation}[theorem]{Motivation}
\newtheorem{remark}[theorem]{Remark}
\newtheorem{TrueFact}[theorem]{True Fact}
\newtheorem{FalseFact}[theorem]{False Fact}
\newtheorem{conjecture}[theorem]{Conjecture}
\newtheorem{conclusion}[theorem]{Conclusion}
\newtheorem{observation}[theorem]{Observation}
\newtheorem{problem}[theorem]{Problem}
\newtheorem{question}[theorem]{Question}
\newtheorem{example}[theorem]{Example}
\newtheorem{note}[theorem]{Note}
\newtheorem{convention}[theorem]{Convention}
\newtheorem{eqn}[theorem]{Equation}
\newtheorem{strategy}[theorem]{Strategy}
\newtheorem{properties}[theorem]{Properties}
\newtheorem{corollary}[theorem]{Corollary}

\newcommand{\HRule}{\rule{\linewidth}{0.5mm}}
\newcommand{\harpvec}[1]{\overrightharp{\ensuremath{\mathbf{#1}}}}
\newcommand*{\threevec}[3]{\ensuremath{\left\langle #1, #2, #3 \right\rangle}}
\newcommand*{\twovec}[2]{\ensuremath{\left\langle #1, #2 \right\rangle}}
\newcommand*{\unitvec}[1]{\ensuremath{\mathbf{\widehat{#1}}}}
\newcommand{\veci}{\ensuremath{\mathbf{\widehat{i}}}}
\newcommand{\vecj}{\ensuremath{\mathbf{\widehat{j}}}}
\newcommand{\veck}{\ensuremath{\mathbf{\widehat{k}}}}
\newcommand{\ds}{\ensuremath{\displaystyle}}


\newcommand{\contrasymb}{\mathrel{\raisebox{.1em}{\reflectbox{\rotatebox[origin=c]{220}{$\lightning$}}}}}

\newenvironment{answer}{\paragraph{Answer.}}{\hfill$\blacklozenge$}
\newenvironment{contraproof}{\paragraph{Proof.}}{\hfill$\contrasymb$}

\begin{document}


\setcounter{chapter}{11}

\chapter{Vectors and the Geometry of Space}

\section{Two and Three Dimensional Space}

\begin{definition}
  Let $\mathbb{R}$ be the collection of real numbers, let $\mathbb{R}^2$ be the
  collection of all \textbf{ordered pairs} of real numbers, and let $\mathbb{R}^3$
  be the collection of all \textbf{ordered triples} of real numbers.

  $\mathbb{R}$ is known as the \textbf{real line}, $\mathbb{R}^2$ is known
  as the \textbf{real plane} or the \textbf{$xy$-plane}, and $\mathbb{R}^3$
  is known as \textbf{real (3D) space} or \textbf{$xyz$-space}.
\end{definition}

\begin{definition}
  The \textbf{distance} between two points $P=(x_1,y_1)$ and
  $Q=(x_2,y_2)$ in $\mathbb{R}^2$ is given by the formula
  \[
    d(P,Q) = \sqrt{(x_2-x_1)^2+(y_2-y_1)^2}
  \]

  The \textbf{distance} between two points $P=(x_1,y_1,z_1)$ and
  $Q=(x_2,y_2,z_2)$ in $\mathbb{R}^3$ is given by the formula
  \[
    d(P,Q) = \sqrt{(x_2-x_1)^2+(y_2-y_1)^2+(z_2-z_1)^2}
  \]
\end{definition}

\begin{problem}
  Plot and find the distance between the following pairs of points:
  \begin{itemize}
    \item $(-2,6)$ and $(3,-6)$
    \item $(0,0,0)$ and $(4,2,4)$
    \item $(3,7,-2)$ and $(-1,7,1)$
    \item $(8,2,1)$ and $(4,-2,7)$
  \end{itemize}
\end{problem}

\vfill

\newpage

\begin{definition}
  \textbf{Simple lines} in $\mathbb{R}^2$ are given by the relations $x=a$,
  and $y=b$ for real numbers $a,b$.

  \textbf{Simple planes} in $\mathbb{R}^3$ are given by the relations $x=a$,
  $y=b$, $z=c$ for real numbers $a,b,c$.
\end{definition}

\begin{definition}
  A \textbf{circle} in $\mathbb{R}^2$ is the set of all points a fixed distance
  (called its \textbf{radius}) from a fixed point (called its \textbf{center}).
  For a center $(a,b)$ and radius $r$, the equation for a circle is
  \[
    (x-a)^2+(y-b)^2=r^2
  \]

  A \textbf{sphere} in $\mathbb{R}^3$ is the set of all points a fixed distance
  (called its \textbf{radius}) from a fixed point (called its \textbf{center}).
  For a center $(a,b,c)$ and radius $r$, the equation for a sphere is
  \[
    (x-a)^2+(y-b)^2+(z-c)^2=r^2
  \]
\end{definition}

\begin{question}
  Sketch the following curves and surfaces.
  \begin{itemize}
    \item $x=3$ in the $xy$-plane and $xyz$-space.
    \item $y=-1$ in the $xy$-plane and $xyz$-space.
    \item $z=0$ in $xyz$-space.
    \item $(x-2)^2+(y+1)^2=9$ in the $xy$-plane.
    \item $x^2+y^2+z^2=4$ in $xyz$-space.
    \item $x^2+(y-1)^2+z^2=1$ in $xyz$-space.
  \end{itemize}
\end{question}

\vfill

\noindent Suggested Homework:
Section $12.1$ numbers $4,$ $6,$ $7,$ $8,$ $10,$ $11,$ $12,$ $14,$ $15,$ $16$

\newpage

\section{Vectors}

\begin{definition}[Vector\index{Vector}]
  A \textbf{vector} $\harpvec v$ is a mathematical object that stores a
  \textbf{magnitude} (a nonnegative real number often thought of as length)
  and \textbf{direction}. Two vectors are \textbf{equal} if and only if they
  have the same magnitude and direction.
\end{definition}

\begin{definition}
  The \textbf{zero vector} $\harpvec0$ has zero magnitude and no direction.
  (This is the only vector without a direction.)
\end{definition}

\begin{definition}
  For a given point $P=(a,b)$ in $\mathbb{R}^2$, its \textbf{position vector}
  is given by $\harpvec{P}=\<a,b\>$: the vector from the origin $(0,0)$ to the
  point $P=(a,b)$.

  For a given point $P=(a,b,c)$ in $\mathbb{R}^3$, its \textbf{position vector}
  is given by $\harpvec{P}=\<a,b,c\>$: the vector from the origin $(0,0,0)$ to
  the point $P=(a,b,c)$.
\end{definition}

\begin{theorem}
  Two vectors are equal if and only if they share the same magnitude and
  direction as a common position vector.
\end{theorem}

\begin{definition}
  Since all vectors are equal to some position vector $\<a,b\>$ or $\<a,b,c\>$,
  we usually define vectors by a position vector written in this
  \textbf{component form}.
  Since the component form of a vector stores the same information as a point,
  we will use both interchangeably, that is, $\<a,b\>=(a,b)\in\mathbb{R}^2$ and
  $\<a,b,c\>=(a,b,c)\in\mathbb{R}^3$
  (although we usually sketch them differently).
\end{definition}

\begin{problem}
  Plot the following points and position vectors.
  \begin{itemize}
    \item $(1,3)$ and $\<1,3\>$ in the $xy$-plane.
    \item $(-2,5)$ and $\<-2,5\>$ in the $xy$-plane.
    \item $(1,1,-3)$ and $\<1,1,-3\>$ in $xyz$-space.
    \item $(0,5,0)$ and $\<0,5,0\>$ in $xyz$-space.
  \end{itemize}
\end{problem}

\vfill

\newpage

\begin{definition}
  Let $P = \left(x_1,y_1,z_1\right)$ and $Q = \left(x_2,y_2,z_2\right).$
  Then the vector with initial point $P$ and terminal point $Q$ is defined as
  \[
    \harpvec{PQ} = \threevec{x_2 - x_1}{y_2 - y_1}{z_2 - z_1}
  \]
\end{definition}

\begin{problem}
  Plot and sketch the points $P$, $Q$ and the vector $\harpvec{PQ}$ for each.
  \begin{itemize}
    \item $P=(1,3)$, $Q=(-3,6)$ in the $xy$-plane
    \item $P=(3,1)$, $Q=(0,-2)$ in the $xy$-plane
    \item $P=(1,1,1)$, $Q=(-3,-1,3)$ in $xyz$-space
    \item $P=(-2,0,3)$, $Q=(1,3,-3)$ in $xyz$-space
  \end{itemize}
\end{problem}

\vfill

\begin{definition}
  The magnitude $|\harpvec{v}|$ of a vector $\harpvec{v}$ in $\mathbb{R}^2$ or
  $\mathbb{R}^3$ is the distance between its initial and terminal points.
\end{definition}

\begin{theorem}
  The magnitude of $\harpvec{v}=\<a,b\>$ is given by
    \[|\harpvec{v}|=\sqrt{a^2+b^2}\]

  The magnitude of $\harpvec{v}=\<a,b,c\>$ is given by
    \[|\harpvec{v}|=\sqrt{a^2+b^2+c^2}\]
\end{theorem}

\newpage

\begin{problem}
  Evaluate the magnitude of the following vectors:
  \begin{itemize}
    \item $\<5,5\>$
    \item $\<-4,3\>$
    \item $\<12,-5\>$
    \item $\<3,1,-2\>$
    \item $\<4,-2,-4\>$
    \item $\<8,0,-6\>$
  \end{itemize}
\end{problem}

\vfill

\subsection{Basic Vector Operations}

\begin{definition}
  \textbf{Vector addition} is defined component-wise as follows for
  $\mathbb{R}^2$ and $\mathbb{R}^3$
  \[
    \harpvec{u}+\harpvec{v}
      =
    \<u_1,u_2\>+\<v_1,v_2\>
      =
    \<u_1+v_1,u_2+v_2\>
  \]
  \[
    \harpvec{u}+\harpvec{v}
      =
    \<u_1,u_2,u_3\>+\<v_1,v_2,v_3\>
      =
    \<u_1+v_1,u_2+v_2,u_3+v_3\>
  \]
\end{definition}

\begin{definition}
  A \textbf{scalar} is simply a real number by itself
  (as opposed to a vector of real numbers).
\end{definition}

\begin{definition}
  \textbf{Scalar multiplication of a vector} is defined component-wise as
  follows for $\mathbb{R}^2$ and $\mathbb{R}^3$:
  \[
    k\harpvec{u}
      =
    k\<u_1,u_2\>
      =
    \<ku_1,ku_2\>
  \]
  \[
    k\harpvec{u}
      =
    k\<u_1,u_2,u_3\>
      =
    \<ku_1,ku_2,ku_3\>
  \]
\end{definition}

\newpage

\begin{problem}
  Sketch the following vectors.
  \begin{itemize}
    \item $\harpvec{u}=\<1,-3\>$, $\harpvec{v}=\<3,1\>$
          and $\harpvec{u}+\harpvec{v}$ in the $xy$-plane.
    \item $\harpvec{u}=\<2,0,1\>$, $\harpvec{v}=\<-2,4,2\>$
          and $\harpvec{u}+\harpvec{v}$ in $xyz$-space.
    \item $\harpvec{u}=\<8,-2\>$ and $\frac{1}{2}\harpvec{u}$ in the $xy$-plane.
    \item $\harpvec{u}=\<5,3,-1\>$ and $3\harpvec{u}$ in $xyz$-space.
  \end{itemize}
\end{problem}

\vfill

\begin{definition}
  A vector $\harpvec{v}$ is a \textbf{unit vector} if $|\harpvec v|=1$.
\end{definition}

\begin{theorem}
  For any non-zero vector $\harpvec{v}$, the vector
  \[
    \frac{1}{|\harpvec v|}\harpvec{v} = \frac{\harpvec v}{|\harpvec v|}
  \]
  is a unit vector.
\end{theorem}

\begin{definition}
  The \textbf{direction} of a vector $\harpvec v$ is the unit vector
  $\frac{\harpvec v}{|\harpvec v|}$.
\end{definition}

\begin{theorem}
  Any vector $\harpvec v$ is the scalar product of its magnitude and direction:
  \[
    \harpvec v = |\harpvec v|\frac{\harpvec v}{|\harpvec v|}
  \]
\end{theorem}

\newpage

\begin{problem}
  Write the following vectors as the scalar product of their magnitude and
  direction:
  \begin{itemize}
    \item $\<5,5\>$
    \item $\<-4,3\>$
    \item $\<12,-5\>$
    \item $\<3,1,-2\>$
    \item $\<4,-2,-4\>$
    \item $\<8,0,-6\>$
  \end{itemize}
\end{problem}

\vfill

\begin{definition}
The \textbf{standard unit vectors} in $\mathbb{R}^2$ are
$\veci=\<1,0\>$ and $\vecj=\<0,1\>$, and any vector in $\mathbb{R}^2$
can be expressed in \textbf{standard unit vector form}:
  \[\<a,b\>=a\veci+b\vecj\]

The \textbf{standard unit vectors} in $\mathbb{R}^3$ are
$\veci=\<1,0,0\>$, $\vecj=\<0,1,0\>$, and $\veck=\<0,0,1\>$, and any vector in
$\mathbb{R}^3$ can be expressed in \textbf{standard unit vector form}:
  \[\<a,b,c\>=a\veci+b\vecj+c\veck\]
\end{definition}

\begin{note}
  Since the $xy$-plane is the the plane $z=0$ in $xyz$-space, we say the
  points $(a,b)=(a,b,0)$ and vectors $\<a,b\>=\<a,b,0\>=a\veci+b\vecj+0\veck$
  are equal.
\end{note}

\newpage

\begin{problem}
  Write the following vectors in standard unit vector form.
  \begin{itemize}
    \item $\<5,5\>$
    \item $\<-4,3\>$
    \item $\<12,-5\>$
    \item $\<3,1,-2\>$
    \item $\<4,-2,-4\>$
    \item $\<8,0,-6\>$
  \end{itemize}
\end{problem}

\vfill

\begin{theorem}
The following properties hold for any two vectors $\harpvec{u}$, $\harpvec{v}$
and scalars $a$, $b$.
  \begin{itemize}
  \item $\harpvec{u}+\harpvec{v} = \harpvec{v}+\harpvec{u}$
  \item $(\harpvec{u}+\harpvec{v})+\harpvec{w} = \harpvec{u}+(\harpvec{v}+\harpvec{w})$
  \item $\harpvec{u}+\harpvec{0} = \harpvec{u}$
  \item $\harpvec{u}+(-\harpvec{u}) = \harpvec{0}$
  \item $0\harpvec{u} = \harpvec{0}$
  \item $1\harpvec{u} = \harpvec{u}$
  \item $a(b\harpvec{u}) = (ab)\harpvec{u}$
  \item $a(\harpvec{u} + \harpvec{v}) = a\harpvec{u} + a\harpvec{v}$
  \item $(a+b)\harpvec{u} = a\harpvec{u} + b\harpvec{u}$
  \end{itemize}
\end{theorem}

\begin{definition}
  \textbf{Vector subtraction} is defined as the addition of a negative:
  \[
    \harpvec{u}-\harpvec{v}
      =
    \harpvec{u}+(-\harpvec{v})
      =
    \<u_1-v_1,u_2-v_2\>
  \]
  \[
    \harpvec{u}-\harpvec{v}
      =
    \harpvec{u}+(-\harpvec{v})
      =
    \<u_1-v_1,u_2-v_2,u_3-v_3\>
  \]
\end{definition}

\noindent Suggested Homework:
Section $12.2$ numbers $3,$ $5,$ $13,$ $14,$ $15,$ $19,$ $21,$ $24,$ $26$

\newpage

\section{The Dot Product}

\begin{definition}
  Let $\theta$ be the angle between two non-zero vectors $\harpvec{u}$, $\harpvec{v}$.
  The \textbf{dot product} $\harpvec{u}\cdot\harpvec{v}$ is the product of their
  lengths when projected into the same direction, obtained by this formula:
  \[
    \harpvec{u} \cdot \harpvec{v} = |\harpvec{u}||\harpvec{v}|\cos \theta
  \]
\end{definition}

\begin{definition}
  The dot product with a zero vector is always zero:
  \[\harpvec{v}\cdot\harpvec{0}=\harpvec{0}\cdot\harpvec{v}=0\]
\end{definition}

\begin{theorem}
By the Law of Cosines:
  \[
    \harpvec{u} \cdot \harpvec{v}
      =
    \<u_1,u_2\>\cdot\<v_1,v_2\>
      =
    u_1v_1 + u_2v_2
  \]
  \[
    \harpvec{u} \cdot \harpvec{v}
      =
    \<u_1,u_2,u_3\>\cdot\<v_1,v_2,v_3\>
      =
    u_1v_1 + u_2v_2 + u_3v_3
  \]
\end{theorem}

\begin{definition}
  Two vectors $\harpvec{u},\harpvec{v}$ are \textbf{orthogonal} if
  $\harpvec{u} \cdot \harpvec{v} = 0$.
\end{definition}

\begin{theorem}
  Two non-zero vectors are orthogonal if the angle $\theta$ between them
  is $\frac{\pi}{2}$ radians.
\end{theorem}

\begin{theorem}
The following properties hold for any three vectors $\harpvec{u}$, $\harpvec{v}$,
$\harpvec{w}$ and scalar $c$.
  \begin{itemize}
  \item $\harpvec{u} \cdot \harpvec{v} = \harpvec{v}\cdot\harpvec{u}$
  \item $(c\harpvec{u})\cdot \harpvec{v} = \harpvec{u} \cdot (c\harpvec{v}) = c(\harpvec{u} \cdot \harpvec{v})$
  \item $\harpvec{u} \cdot (\harpvec{v} + \harpvec{w}) = \harpvec{u}\cdot\harpvec{v} + \harpvec{u}\cdot \harpvec{w}$
  \item $\harpvec{u} \cdot \harpvec{u} = |\harpvec{u}|^2$
  \end{itemize}
\end{theorem}

\begin{problem}
  Solve for $\cos\theta$ for the following pairs of vectors.
  \begin{itemize}
    \item $\harpvec{u}=\<4,-3\>$\\ $\harpvec{v}=\<5,12\>$
    \item $\harpvec{u}=\<1,4,2\>$\\ $\harpvec{v}=\<4,1,-2\>$
    \item $\harpvec{u}=\<0,5,-11\>$\\ $\harpvec{v}=\<2,0,0\>$
  \end{itemize}
\end{problem}

\vfill

\begin{definition}
  The work $W$ done by a force vector $\harpvec{F}$ over a displacement
  vector $\harpvec{D}$ is given by
  \[
    W = \harpvec{F} \cdot \harpvec{D}
      =
    |\harpvec{F}||\harpvec{D}|\cos \theta
  \]
\end{definition}

\noindent Suggested Homework: Section $12.3$ numbers
$3,$ $5,$ $6,$ $7,$ $8,$ $9,$ $10,$ $11,$ $15,$ $17,$ $21,$ $27,$ $41,$ $42,$ $44$


\newpage


\section{The Cross Product}

  \begin{definition}
    For any two non-parallel vectors $\harpvec{u}$, $\harpvec{v}$ in $\mathbb R^3$,
    the \textbf{Right-Hand Rule} gives a specific direction orthogonal to both:
    position $\harpvec{u}$ with your right thumb and $\harpvec{v}$ with your
    right index finger, and let your middle finger extend orthogonal to both
    to give this direction.
  \end{definition}

  \begin{definition}
    Let $\theta$ be the angle between two non-zero vectors $\harpvec{u}$,
    $\harpvec{v}$ in $\mathbb{R}^3$, and let $\harpvec{n}$ be the direction
    given by the Right-Hand Rule.
    The \textbf{cross product} $\harpvec{u}\times\harpvec{v}$ is the vector
    orthogonal to both which follows the Right-Hand Rule and has magnitude
    equal to the area of the parallelogram formed from both.
    \[
      \harpvec{u}\times\harpvec{v}
        =
      (|\harpvec{u}||\harpvec{v}|\sin\theta)\harpvec{n}
    \]
    \[
      |\harpvec{u}\times\harpvec{v}|
        =
      |\harpvec{u}||\harpvec{v}|\sin\theta
    \]
  \end{definition}

  \begin{definition}
    The cross product with a zero vector is always the zero vector:
    \[\harpvec{v}\times\harpvec{0}=\harpvec{0}\times\harpvec{v}=\harpvec{0}\]
  \end{definition}

  \begin{theorem}
    The following properties hold for any three vectors $\harpvec{u}$, $\harpvec{v}$,
    $\harpvec{w}$ and scalars $a$,$b$.
    \begin{itemize}
    \item $(a\harpvec{u}) \times (b\harpvec{v}) = (ab)(\harpvec{u} \times \harpvec{v})$
    \item
      $\harpvec{u} \times (\harpvec{v} + \harpvec{w}) =
      \harpvec{u} \times \harpvec{v} + \harpvec{u} \times \harpvec{w}$
    \item
      $(\harpvec{v} + \harpvec{w}) \times \harpvec{u} =
      \harpvec{v} \times \harpvec{u} + \harpvec{w} \times \harpvec{u}$
    \item $\harpvec{v} \times \harpvec{u} = -(\harpvec{u} \times \harpvec{v})$
    \end{itemize}
  \end{theorem}

  \begin{definition}
    Two vectors $\harpvec{u},\harpvec{v}$ are \textbf{parallel} if
    $\harpvec{u} \times \harpvec{v} = 0$.
  \end{definition}

  \begin{theorem}
    Two non-zero vectors are parallel if the angle $\theta$ between them
    is $0$ or $\pi$ radians.
  \end{theorem}

  \begin{definition}
    The cross products of the standard unit vectors are given as follows:
    \begin{itemize}
      \item $\veci \times \vecj = \veck$
      \item $\vecj \times \veck = \veci$
      \item $\veck \times \veci = \vecj$
    \end{itemize}
  \end{definition}

  \begin{definition}
    A \textbf{determinant} is a short hand for writing certain commonly
    occuring algebraic expressions:
      \[
        \begin{array}{|c c|}
        a_1 & a_2 \\
        b_1 & b_2 \\
        \end{array}
          =
        a_1b_2 - a_2b_1
      \]
      \[
        \begin{array}{|c c c|}
        a_1 & a_2 & a_3 \\
        b_1 & b_2 & b_3 \\
        c_1 & c_2 & c_3 \\
        \end{array}
          =
        a_1 \,
        \begin{array}{|c c|}
        b_2 & b_3 \\
        c_2 & c_3 \\
        \end{array}
          -
        a_2 \,
        \begin{array}{|c c|}
        b_1 & b_3 \\
        c_1 & c_3 \\
        \end{array}
          +
        a_3 \,
        \begin{array}{|c c|}
        b_1 & b_2 \\
        c_1 & c_2 \\
        \end{array}
      \]
  \end{definition}

\newpage

  \begin{theorem}
    By breaking up $\harpvec{u}$, $\harpvec{v}$ into standard unit vectors:
    \[
    \harpvec{u} \times \harpvec{v}
      =
    \begin{array}{|c c c|}
    \veci & \vecj & \veck \\
    u_1 & u_2 & u_3 \\
    v_1 & v_2 & v_3 \\
    \end{array}
      =
    \left\<
      \begin{array}{|c c|}
      u_2 & u_3 \\
      v_2 & v_3 \\
      \end{array}
        \,,-\,
      \begin{array}{|c c|}
      u_1 & u_3 \\
      v_1 & v_3 \\
      \end{array}
        \,,\,
      \begin{array}{|c c|}
      u_1 & u_2 \\
      v_1 & v_2 \\
      \end{array}
    \right\>
    \]
  \end{theorem}

\begin{problem}
  Use the cross product to find a vector normal to both $\harpvec{u}$
  and $\harpvec{v}$.
    \begin{itemize}
      \item $\harpvec{u}=\<4,-3,0\>$\\ $\harpvec{v}=\<2,6,-3\>$
      \item $\harpvec{u}=\<1,4,2\>$\\ $\harpvec{v}=\<4,1,-2\>$
      \item $\harpvec{u}=\<0,5,-11\>$\\ $\harpvec{v}=\<2,0,0\>$
    \end{itemize}
\end{problem}

\vfill

\begin{definition}
  The torque $\tau$ done by a force vector $\harpvec{F}$ on an arm given by
  $\harpvec{D}$ is given by
  \[
    \tau = |\harpvec{F} \times \harpvec{D}|
      =
    |\harpvec{F}||\harpvec{D}|\sin \theta
  \]
\end{definition}

\begin{theorem}
  The volume of a parallelpiped determined by the vectors
  $\harpvec{u}$, $\harpvec{v}$, $\harpvec{w}$, is given by
  the \textbf{triple scalar product}
    \[
      (\harpvec{u}\times\harpvec{v})\cdot\harpvec{w} =
      \begin{array}{|c c c|}
      u_1 & u_2 & u_3 \\
      v_1 & v_2 & v_3 \\
      w_1 & w_2 & w_3 \\
      \end{array}
    \]
\end{theorem}

\noindent Suggested Homework:
Section $12.4$ numbers $1 - 3,$ $17,$ $19,$ $28,$ $29,$ $33,$ $35$

% % \newpage

% % \setcounter{section}{4}

% % \section{Equations in 3-Space}

% % \begin{eqn}[Parametrization of a Line\index{Parametrization of a Line}]
% % Let $O = \left(0,0,0\right)$ be the origin in $\mathbb{R}^3, P_0 = \left(x_0,y_0,z_0\right)$ be a point in $\mathbb{R}^3,$ and $\harpvec{v} = \threevec{A}{B}{C}$ be a vector in $\mathbb{R}^3$ parallel to the line being parametrized.  Then the line through $P_0$ parallel to $\harpvec{v}$ is $$\harpvec{r}(t) = \harpvec{OP_0} + t\harpvec{v} \hspace{.5in} t \in \mathbb{R}.$$ This can also be written as $$x = x_0 + At, \hspace{.5in} y = y_0 + Bt, \hspace{.5in} z = z_0 + Ct \hspace{.5in} t \in \mathbb{R}.$$ or as the symmetric equation $$\frac{x-x_0}{A} = \frac{y-y_o}{B} = \frac{z-z_0}{C}.$$
% % \end{eqn}

% % \begin{eqn}[Parametrization of a Line Segment\index{Parametrization of Line Segment}]
% % Let $O$ denote the origin, $P$ be the initial point of a line segment, and $Q$ be the terminal point of a line segment.  Then the line segment $\overline{PQ}$ can be parametrized as $$\harpvec{r}(t) = (1-t)\harpvec{OP} + t\harpvec{OQ} \hspace{.5in} 0 \leq t \leq 1.$$
% % \end{eqn}

% % \newpage

% % \begin{problem}
% % Find a vector equation and parametric equation for the line that passes through the point $(5,1,3)$ and is parallel to the vector $\threevec{1}{4}{-2}.$
% % \end{problem}

% % \vfill

% % \begin{problem}
% % Find the parametric Equation of the line segment from $(2,4,-3)$ to $(3,-1,1).$
% % \end{problem}

% % \vfill

% % \begin{eqn}[Planes\index{Plane (Equation)}]
% % Let $P_0 = \left(x_0,y_0,z_0\right)$ be a point in the plane and $\harpvec{n} = \threevec{a}{b}{c}$ be a vector normal to the plane.  Then the equation of the plane is $$a\left(x-x_0\right) + b\left(y-y_0\right) + c\left(z-z_0\right).$$
% % \end{eqn}\label{EquationOfPlane}

% % \vfill

% % \noindent Suggested Homework: Section $12.5$ numbers $3,$ $4,$ $6,$ $7,$ $17,$ $19,$ $24,$ $27,$ $31,$ $32$

\end{document}