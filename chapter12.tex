\documentclass[letterpaper, twoside, 12pt]{book}

\usepackage{amsmath,amsfonts, amsthm}
\usepackage{yfonts}
\usepackage{amsrefs}
\usepackage{fancyhdr}
\usepackage{graphicx}
\usepackage{float}
\usepackage[margin=1in]{geometry}
\usepackage{array}
\usepackage{esint}
\usepackage{harpoon}
\usepackage{stmaryrd}
\usepackage{multicol}
\usepackage{multirow}
\usepackage{pgf,tikz}
\usetikzlibrary{arrows}

\usepackage{makeidx}
\makeindex

\definecolor{AuburnOrange}{RGB}{221,85,12}
\definecolor{AuburnBlue}{RGB}{3,36,77}
\definecolor{AuburnSecondaryBlue}{RGB}{73,110,156}
\definecolor{AuburnSecondaryOrange}{RGB}{246,128,38}
\definecolor{AlabamaCrimson}{RGB}{163,38,65}
\definecolor{LSUpurple}{RGB}{70,29,124}
\definecolor{VanderbiltGold}{RGB}{207,181,59}

\usepackage[pdfpagelabels]{hyperref}
\hypersetup{colorlinks=true,linkcolor=AuburnSecondaryOrange}

% \usepackage[tracking]{microtype}
% \UseMicrotypeSet{all}
% \SetTracking[spacing = {35*,0*,0*}]{encoding = *}{7}
% \linespread{1.025}

\def\scaleint#1{\vcenter{\hbox{\scaleto[3ex]{\displaystyle\int}{#1}}}}
\def\bs{\!\!}

\renewcommand{\arraystretch}{1.5}

\newcommand{\<}{\langle}
\renewcommand{\>}{\rangle}

\pagestyle{fancy} \headheight 14.49998pt

\newcommand{\tstamp}{\today}
\renewcommand{\chaptermark}[1]{\markboth{#1}{}}
\renewcommand{\chaptermark}[1]{\markright{#1}}

\lhead[\fancyplain{}{Clontz \thepage}]         {\fancyplain{}{\scshape\nouppercase{\rightmark}}}

\chead[\fancyplain{}{}]
{\fancyplain{}{}}


\rhead[\fancyplain{}{Calculus II Lecture Notes}]       {\fancyplain{}{Clontz \thepage}}

\lfoot[\fancyplain{}{Auburn University}]                 {\fancyplain{\tstamp}{\tstamp}}

\cfoot[\fancyplain{\thepage}{}]         {\fancyplain{\thepage}{}}

\rfoot[\fancyplain{\tstamp} {\tstamp}]  {\fancyplain{}{Auburn University}}

\theoremstyle{definition}
\newtheorem{theorem}{Theorem}

% \theoremstyle{plain}
\newtheorem{proposition}[theorem]{Proposition}
\newtheorem{recall}[theorem]{Recall}

\theoremstyle{definition}
\newtheorem{definition}[theorem]{Definition}
\newtheorem{notation}[theorem]{Notation}
\newtheorem{goal}[theorem]{Goal}
\newtheorem{motivation}[theorem]{Motivation}
\newtheorem{remark}[theorem]{Remark}
\newtheorem{TrueFact}[theorem]{True Fact}
\newtheorem{FalseFact}[theorem]{False Fact}
\newtheorem{conjecture}[theorem]{Conjecture}
\newtheorem{conclusion}[theorem]{Conclusion}
\newtheorem{observation}[theorem]{Observation}
\newtheorem{problem}[theorem]{Problem}
\newtheorem{question}[theorem]{Question}
\newtheorem{example}[theorem]{Example}
\newtheorem{note}[theorem]{Note}
\newtheorem{convention}[theorem]{Convention}
\newtheorem{eqn}[theorem]{Equation}
\newtheorem{strategy}[theorem]{Strategy}
\newtheorem{properties}[theorem]{Properties}
\newtheorem{corollary}[theorem]{Corollary}

\newcommand{\HRule}{\rule{\linewidth}{0.5mm}}
\newcommand{\harpvec}[1]{\overrightharp{\ensuremath{\mathbf{#1}}}}
\newcommand*{\threevec}[3]{\ensuremath{\left\langle #1, #2, #3 \right\rangle}}
\newcommand*{\twovec}[2]{\ensuremath{\left\langle #1, #2 \right\rangle}}
\newcommand*{\unitvec}[1]{\ensuremath{\mathbf{\widehat{#1}}}}
\newcommand{\veci}{\ensuremath{\mathbf{\widehat{i}}}}
\newcommand{\vecj}{\ensuremath{\mathbf{\widehat{j}}}}
\newcommand{\veck}{\ensuremath{\mathbf{\widehat{k}}}}
\newcommand{\ds}{\ensuremath{\displaystyle}}


\newcommand{\contrasymb}{\mathrel{\raisebox{.1em}{\reflectbox{\rotatebox[origin=c]{220}{$\lightning$}}}}}

\newenvironment{answer}{\paragraph{Answer.}}{\hfill$\blacklozenge$}
\newenvironment{contraproof}{\paragraph{Proof.}}{\hfill$\contrasymb$}

\begin{document}


\setcounter{chapter}{11}

\chapter{Vectors and the Geometry of Space}

\section{Two and Three Dimensional Space}

\begin{definition}
  Let $\mathbb{R}$ be the collection of real numbers, let $\mathbb{R}^2$ be the
  collection of all \textbf{ordered pairs} of real numbers, and let $\mathbb{R}^3$
  be the collection of all \textbf{ordered triples} of real numbers.

  $\mathbb{R}$ is known as the \textbf{real line}, $\mathbb{R}^2$ is known
  as the \textbf{real plane} or the \textbf{$xy$-plane}, and $\mathbb{R}^3$
  is known as \textbf{real (3D) space} or \textbf{$xyz$-space}.
\end{definition}

\begin{definition}
  The \textbf{distance} between two points $P=(x_1,y_1)$ and
  $Q=(x_2,y_2)$ in $\mathbb{R}^2$ is given by the formula
  \[
    d(P,Q) = \sqrt{(x_2-x_1)^2+(y_2-y_1)^2}
  \]

  The \textbf{distance} between two points $P=(x_1,y_1,z_1)$ and
  $Q=(x_2,y_2,z_2)$ in $\mathbb{R}^3$ is given by the formula
  \[
    d(P,Q) = \sqrt{(x_2-x_1)^2+(y_2-y_1)^2+(z_2-z_1)^2}
  \]
\end{definition}

\begin{problem}
  Plot and find the distance between the following pairs of points:
  \begin{itemize}
    \item $(-2,6)$ and $(3,-6)$
    \item $(0,0,0)$ and $(4,2,4)$
    \item $(3,7,-2)$ and $(-1,7,1)$
    \item $(8,2,1)$ and $(4,-2,7)$
  \end{itemize}
\end{problem}

\vfill

\newpage

\begin{definition}
  \textbf{Simple lines} in $\mathbb{R}^2$ are given by the relations $x=a$,
  and $y=b$ for real numbers $a,b$.

  \textbf{Simple planes} in $\mathbb{R}^3$ are given by the relations $x=a$,
  $y=b$, $z=c$ for real numbers $a,b,c$.
\end{definition}

\begin{definition}
  A \textbf{circle} in $\mathbb{R}^2$ is the set of all points a fixed distance
  (called its \textbf{radius}) from a fixed point (called its \textbf{center}).
  For a center $(a,b)$ and radius $r$, the equation for a circle is
  \[
    (x-a)^2+(y-b)^2=r^2
  \]

  A \textbf{sphere} in $\mathbb{R}^3$ is the set of all points a fixed distance
  (called its \textbf{radius}) from a fixed point (called its \textbf{center}).
  For a center $(a,b,c)$ and radius $r$, the equation for a sphere is
  \[
    (x-a)^2+(y-b)^2+(z-c)^2=r^2
  \]
\end{definition}

\begin{question}
  Sketch the following curves and surfaces.
  \begin{itemize}
    \item $x=3$ in the $xy$-plane and $xyz$-space.
    \item $y=-1$ in the $xy$-plane and $xyz$-space.
    \item $z=0$ in $xyz$-space.
    \item $(x-2)^2+(y+1)^2=9$ in the $xy$-plane.
    \item $x^2+y^2+z^2=4$ in $xyz$-space.
    \item $x^2+(y-1)^2+z^2=1$ in $xyz$-space.
  \end{itemize}
\end{question}

\vfill

\noindent Suggested Homework:
Section $12.1$ numbers $4,$ $6,$ $7,$ $8,$ $10,$ $11,$ $12,$ $14,$ $15,$ $16$

% \newpage

% \section{Vectors}

% \begin{definition}[Vector\index{Vector}]
%   A \textbf{vector} is a mathematical object that stores a \textbf{magnitude}
%   (often thought of as length) and \textbf{direction}.
%   Two vectors are \textbf{equal} if and only if they have the same magnitude and
%   direction.
% \end{definition}

% \begin{definition}
%   For a given point $P=(a,b)$ in $\mathbb{R}^2$, its \textbf{position vector}
%   is given by $\harpvec{P}=\<a,b\>$: the vector from the origin $(0,0)$ to the
%   point $P=(a,b)$.

%   For a given point $P=(a,b,c)$ in $\mathbb{R}^3$, its \textbf{position vector}
%   is given by $\harpvec{P}=\<a,b,c\>$: the vector from the origin $(0,0,0)$ to
%   the point $P=(a,b,c)$.
% \end{definition}

% \begin{theorem}
%   Two vectors are equal if and only if they share the same magnitude and
%   direction as a common position vector.
% \end{theorem}

% \begin{definition}
%   Since all vectors are equal to some position vector $\<a,b\>$ or $\<a,b,c\>$,
%   we usually define vectors by a position vector written in this
%   \textbf{component form}.
%   Since the component form of a vector stores the same information as a point,
%   we will use both interchangeably, that is, $\<a,b\>=(a,b)\in\mathbb{R}^2$ and
%   $\<a,b,c\>=(a,b,c)\in\mathbb{R}^3$
%   (although we usually sketch them differently).
% \end{definition}

% \begin{problem}
%   Plot the following points and position vectors.
%   \begin{itemize}
%     \item $(1,3)$ and $\<1,3\>$ in the $xy$-plane.
%     \item $(-2,5)$ and $\<-2,5\>$ in the $xy$-plane.
%     \item $(1,1,-3)$ and $\<1,1,-3\>$ in $xyz$-space.
%     \item $(0,5,0)$ and $\<0,5,0\>$ in $xyz$-space.
%   \end{itemize}
% \end{problem}

% \vfill

% \newpage

% \begin{definition}
%   Let $P = \left(x_1,y_1,z_1\right)$ and $Q = \left(x_2,y_2,z_2\right).$
%   Then the vector with initial point $P$ and terminal point $Q$ is defined as
%   \[
%     \harpvec{PQ} = \threevec{x_2 - x_1}{y_2 - y_1}{z_2 - z_1}
%   \]
% \end{definition}

% \begin{problem}
%   Plot and sketch the points $P$, $Q$ and the vector $\harpvec{PQ}$ for each.
%   \begin{itemize}
%     \item $P=(1,3)$, $Q=(-3,6)$ in the $xy$-plane
%     \item $P=(-2,0,3)$, $Q=(1,3,-3)$ in $xyz$-space
%   \end{itemize}
% \end{problem}

% \vfill

% \begin{definition}
%   The magnitude of a vector in $\mathbb{R}^2$ or $\mathbb{R}^3$ is the distance
%   between its initial and terminal points.
% \end{definition}

% \begin{theorem}
%   The magnitude of $\<a,b\>$ is given by $\sqrt{a^2+b^2}$, and the
%   magnitude of $\<a,b,c\>$ is given by $\sqrt{a^2+b^2+c^2}$.
% \end{theorem}

% \begin{problem}
%   Give the magitude of $\harpvec{PQ}$ for each bullet in the previous problem.
% \end{problem}

% \vfill

% \newpage

% \subsection{Operations}

% \begin{definition}
%   \textbf{Vector addition} is defined component-wise as follows for
%   $\mathbb{R}^2$ and $\mathbb{R}^3$
%   \[
%     \harpvec{u}+\harpvec{v}
%       =
%     \<u_1,u_2\>+\<v_1,v_2\>
%       =
%     \<u_1+v_1,u_2+v_2\>
%   \]
%   \[
%     \harpvec{u}+\harpvec{v}
%       =
%     \<u_1,u_2,u_3\>+\<v_1,v_2,v_3\>
%       =
%     \<u_1+v_1,u_2+v_2,u_3+v_3\>
%   \]
% \end{definition}

% \begin{definition}
%   A \textbf{scalar} is simply a real number by itself
%   (as opposed to a vector of real numbers).
% \end{definition}

% \begin{definition}
%   \textbf{Scalar multiplication of a vector} is defined component-wise as
%   follows for $\mathbb{R}^2$ and $\mathbb{R}^3$:
%   \[
%     k\harpvec{u}
%       =
%     k\<u_1,u_2\>
%       =
%     \<ku_1,ku_2\>
%   \]
%   \[
%     k\harpvec{u}
%       =
%     k\<u_1,u_2,u_3\>
%       =
%     \<ku_1,ku_2,ku_3\>
%   \]
% \end{definition}

% % \begin{definition}[Unit Vector\index{Unit Vector}]
% % A \textbf{unit vector} is a vector whose magnitude is $1.$  Note that we can given a vector $\harpvec{v}$, we can form a unit vector $\unitvec{v}$ by dividing by the magnitude of $\harpvec{v}.$  That is to say, Let $\harpvec{v} = \threevec{v_1}{v_2}{v_3}.$  Then $$\unitvec{v} = \frac{1}{\left|\harpvec{v}\right|}\threevec{v_1}{v_2}{v_3}.$$
% % \end{definition}

% % \begin{definition}[Standard Vectors\index{Standard Vectors}]
% % Any vector can be denoted as the linear combination of the \textbf{standard unit vectors} $\veci = \threevec{1}{0}{0}, \vecj = \threevec{0}{1}{0},$ and $\veck = \threevec{0}{0}{1}.$  So given a vector $\harpvec{v} = \threevec{v_1}{v_2}{v_3},$ one can express it with respect to the standard vectors as $$\harpvec{v} = \threevec{v_1}{v_2}{v_3} = v_1 \veci + v_2\vecj + v_3\veck.$$  This text, however, will more often than not use the angle brace notation.
% % \end{definition}

% % \begin{definition}[Dot Product\index{Dot Product}]
% % Let $\harpvec{u} = \threevec{u_1}{u_2}{u_3}$ and $\harpvec{v} = \threevec{v_1}{v_2}{v_3}$.  Then the dot product or Euclidean Inner Product as it is sometimes referred is $$\harpvec{u} \cdot \harpvec{v} = u_1v_1 + u_2v_2 + u_3v_3 = \left|\harpvec{u}\right|\left|\harpvec{v}\right|\cos\left(\theta\right).$$
% % \end{definition}

% % \begin{theorem}
% % Two nonzero vectors $\harpvec{u}$ and $\harpvec{v}$ are \textbf{orthogonal} if and only if $\harpvec{u}\cdot\harpvec{v} = 0.$
% % \end{theorem}

% % \begin{problem}
% % Show that if two non-zero vector are orthogonal then $\harpvec{u} \cdot \harpvec{v} = 0.$
% % \end{problem}

% % \newpage

% % \begin{definition}[Cross Product\index{Cross Product}]
% % Let $\harpvec{u} = \threevec{u_1}{u_2}{u_3}$ and $\harpvec{v} = \threevec{v_1}{v_2}{v_3}$.  Then the cross product is the determinant of the following matrix:
% % $$\begin{array}{rcl}
% % \harpvec{u} \times \harpvec{v} &=& \left|\begin{array}{ccc}
% % \veci & \vecj & \veck \\
% % u_1 & u_2 & u_3 \\
% % v_1 & v_2 & v_3
% % \end{array}\right|\\[2.5em]
% %  &=& \left|\begin{array}{cc} u_2 & u_3 \\ v_2 & v_3 \end{array}\right|\veci - \left|\begin{array}{cc} u_1 & u_3 \\ v_1 & v_3 \end{array}\right|\vecj + \left|\begin{array}{cc} u_1 & u_2 \\ v_1 & v_2 \end{array}\right|\veck\\
% %  &=& \threevec{u_2v_3 - u_3v_2}{u_3v_1 - u_1v_3}{u_1v_2 - u_2v_1}.
% % \end{array}$$
% % \end{definition}

% % \begin{observation}
% % The cross product of two vectors $\harpvec{u}$ and $\harpvec{v}$ gives us a vector that is orthogonal to both $\harpvec{u}$ and $\harpvec{v}$.
% % \end{observation}

% % \begin{definition}[Equivalent to Cross Product\index{Equivalent to Cross Product}]
% % Let $\harpvec{u} = \threevec{u_1}{u_2}{u_3}$ and $\harpvec{v} = \threevec{v_1}{v_2}{v_3}$.  Then the cross product can also be defined as $$\left|\harpvec{u} \times \harpvec{v}\right| = \left|\harpvec{u}\right|\left|\harpvec{v}\right|\sin\left(\theta\right).$$
% % \end{definition}

% % \begin{problem}
% % Show that is two non-zero vectors $\harpvec{u}$ and $\harpvec{v}$ are parallel if and only if $\harpvec{u} \times \harpvec{v} = \harpvec{0}.$
% % \end{problem}

% % \vfill

% % \noindent Suggested Homework: Section $12.1$ numbers $4,$ $6,$ $7,$ $8,$ $10,$ $11,$ $12,$ $14,$ $15,$ $16$\\
% % Section $12.2$ numbers $3,$ $5,$ $13,$ $14,$ $15,$ $19,$ $21,$ $24,$ $26$\\
% % Section $12.3$ numbers $3,$ $5,$ $6,$ $7,$ $8,$ $9,$ $10,$ $11,$ $15,$ $17,$ $21,$ $27,$ $41,$ $42,$ $44$\\
% % Section $12.4$ numbers $1 - 3,$ $17,$ $19,$ $28,$ $29,$ $33,$ $35$

% % \newpage

% % \setcounter{section}{4}

% % \section{Equations in 3-Space}

% % \begin{eqn}[Parametrization of a Line\index{Parametrization of a Line}]
% % Let $O = \left(0,0,0\right)$ be the origin in $\mathbb{R}^3, P_0 = \left(x_0,y_0,z_0\right)$ be a point in $\mathbb{R}^3,$ and $\harpvec{v} = \threevec{A}{B}{C}$ be a vector in $\mathbb{R}^3$ parallel to the line being parametrized.  Then the line through $P_0$ parallel to $\harpvec{v}$ is $$\harpvec{r}(t) = \harpvec{OP_0} + t\harpvec{v} \hspace{.5in} t \in \mathbb{R}.$$ This can also be written as $$x = x_0 + At, \hspace{.5in} y = y_0 + Bt, \hspace{.5in} z = z_0 + Ct \hspace{.5in} t \in \mathbb{R}.$$ or as the symmetric equation $$\frac{x-x_0}{A} = \frac{y-y_o}{B} = \frac{z-z_0}{C}.$$
% % \end{eqn}

% % \begin{eqn}[Parametrization of a Line Segment\index{Parametrization of Line Segment}]
% % Let $O$ denote the origin, $P$ be the initial point of a line segment, and $Q$ be the terminal point of a line segment.  Then the line segment $\overline{PQ}$ can be parametrized as $$\harpvec{r}(t) = (1-t)\harpvec{OP} + t\harpvec{OQ} \hspace{.5in} 0 \leq t \leq 1.$$
% % \end{eqn}

% % \newpage

% % \begin{problem}
% % Find a vector equation and parametric equation for the line that passes through the point $(5,1,3)$ and is parallel to the vector $\threevec{1}{4}{-2}.$
% % \end{problem}

% % \vfill

% % \begin{problem}
% % Find the parametric Equation of the line segment from $(2,4,-3)$ to $(3,-1,1).$
% % \end{problem}

% % \vfill

% % \begin{eqn}[Planes\index{Plane (Equation)}]
% % Let $P_0 = \left(x_0,y_0,z_0\right)$ be a point in the plane and $\harpvec{n} = \threevec{a}{b}{c}$ be a vector normal to the plane.  Then the equation of the plane is $$a\left(x-x_0\right) + b\left(y-y_0\right) + c\left(z-z_0\right).$$
% % \end{eqn}\label{EquationOfPlane}

% % \vfill

% % \noindent Suggested Homework: Section $12.5$ numbers $3,$ $4,$ $6,$ $7,$ $17,$ $19,$ $24,$ $27,$ $31,$ $32$

\end{document}