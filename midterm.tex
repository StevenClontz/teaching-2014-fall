\documentclass[12pt]{exam}

\newcommand{\ds}{\ensuremath{\displaystyle}}

\usepackage{multicol}
\usepackage{multirow}
\renewcommand{\arraystretch}{1.5}

% \printanswers

\begin{document}

\begin{center}
\fbox{\fbox{\parbox{5.5in}{\centering
Calculus II - Fall 2014 - Mr. Clontz - Midterm Exam
}}}
\end{center}
\vspace{0.1in}
\makebox[\textwidth]{Name:\enspace\hrulefill\space 9am / 10am}

\vspace{12pt}

\begin{itemize}
  \item If you completed the practice midterm, turn it in before beginning
        this exam.
  \item This exam is closed-note and closed-book.
  \item The withdrawal deadline is the evening of Tuesday, October 7.
        If you need me to post your grade to Canvas before the deadline,
        please mark this circle:\\
        $\bigcirc$ POST GRADE BEFORE WITHDRAWAL DEADLINE
\end{itemize}

\noindent
Good luck! Here are the series tests in case you need them:

\vspace{12pt}

{\scriptsize \centering \index{List Of Series Tests}\begin{tabular}{|c|c|c|}
\hline
{\bf Test} & {\bf When to Use} & {\bf Conclusion} \\
\hline

\multirow{2}{*}{{\bf Geometric Series}} & \multirow{2}{*}{$\sum_{k=1}^{\infty} ar^k$} & Converges to $\frac{a}{1-r}$ if $|r| < 1$; \\ & &diverges if $|r| \geq 1$. \\
\hline

{\bf Divergence Test} & All Series & If $\lim_{k \rightarrow \infty} a_k \neq 0$, the series diverges. \\
\hline

\multirow{2}{*}{{\bf Integral Test}} & $\sum_{k=1}^{\infty} a_k$ where $f(k) = a_k$ and & $\sum_{k = 1}^{\infty} a_k$ and $\int_1^\infty f(x) \, dx$ \\
&$f$ is continuous, decreasing, and $f(x) \geq 0$ & {\bf both} converge or {\bf both} diverge.\\
\hline

{\bf $p$-series} & $\sum_{k=1}^{\infty} \frac{1}{k^p}$ & Converges for $p > 1$, diverges for $p \leq 1$.\\
\hline

\multirow{2}{*}{{\bf Comparison Test}} & \multirow{2}{*}{$\sum_{k=1}^{\infty} a_k$ and $\sum_{k=1}^{\infty} b_k$, where $0 \leq a_k \leq b_k$} & If $\sum_{k=1}^{\infty} b_k$ converges, then $\sum_{k=1}^{\infty} a_k$ converges. \\ & & If $\sum_{k=1}^{\infty} a_k$ diverges, then $\sum_{k=1}^{\infty} b_k$ diverges.\\
\hline

\multirow{2}{*}{{\bf Limit Comparison Test}} & $\sum_{k=1}^{\infty} a_k$ and $\sum_{k=1}^{\infty} b_k$, where & $\sum_{k=1}^{\infty} a_k$ and $\sum_{k=1}^{\infty} b_k$ \\ & $a_k, b_k > 0$ and $\lim_{k\rightarrow \infty} \frac{a_k}{b_k} = L > 0$ & {\bf both} converge or {\bf both} diverge. \\
\hline

\multirow{2}{*}{{\bf Alternating Series Test}} & \multirow{2}{*}{$\sum_{k=1}^{\infty} \left(-1\right)^{k+1} a_k$ where $a_k > 0$ for all $k$} & If $\lim_{k \rightarrow \infty} a_k = 0$ and $a_{k+1} \leq a_k$ for all $k$, \\ & & then the series converges. \\
\hline

\multirow{2}{*}{{\bf Absolute Convergence}} & Series with some positive and some & If $\sum_{k=1}^{\infty} |a_k|$ converges, then\\
 & negative terms (including alternating series) & $\sum_{k=1}^{\infty} a_k$ converges (absolutely).\\
 \hline

\multirow{4}{*}{{\bf Ratio Test}} & & For $\lim_{k \rightarrow \infty} \left|\frac{a_{k+1}}{a_k}\right| = L,$ \\ & \multirow{2}{*}{Any Series (especially those involving} & if $L < 1$, $\sum_{k = 1}^{\infty} a_k$ converges absolutely, \\ &exponentials and/or factorials) & if $L > 1$, $\sum_{k=1}^{\infty} a_k$ diverges, \\ & & if $L = 1$, no conclusion.\\
\hline

\multirow{4}{*}{{\bf Root Test}} & & For $\lim_{k \rightarrow \infty} \sqrt[k]{|a_k|} = L,$ \\ & \multirow{2}{*}{Any Series (especially those involving} & if $L < 1$, $\sum_{k = 1}^{\infty} a_k$ converges absolutely, \\ &exponentials) & if $L > 1$, $\sum_{k=1}^{\infty} a_k$ diverges, \\ & & if $L = 1$, no conclusion.\\
\hline
\end{tabular} }

\newpage

\begin{center}
  \textbf{Multiple Choice (10 points total)}
\end{center}
\noindent
Please only mark the correct choice for each question.

\begin{questions}

\setcounter{question}{0}
\question[3]
Nick Saban wrote the following\footnote{
  I can't back that up, but I feel like he would, y'know?
}:

\begin{center}
``Since $\ds\lim_{n\to\infty} \frac{n}{n^2+1} = 0$, the series
$\ds\sum_{n=0}^\infty \frac{n}{n^2+1}$ converges.''
\end{center}

Why is this horribly wrong?

\begin{checkboxes}
\choice The limit $\lim_{n\to\infty} \frac{n}{n^2+1}$ is
        $\frac{1}{2}$, not $0$.
\choice Since $\lim_{n\to\infty} \frac{n}{n^2+1} = 0$, the series
        $\sum_{n=0}^\infty \frac{n}{n^2+1}$ diverges.
\choice The Divergence Test requires that the limit be different from $0$, and
        cannot prove that a series converges.
\choice The Divergence Test doesn't work on a series with only positive terms.
\end{checkboxes}

\vfill

\question[3]
Integration by parts is the reverse version of which rule?

\begin{checkboxes}
\choice Chain Rule ...... $\frac{d}{dx}[f(g(x))]=f'(g(x))g'(x)$
\choice Power Rule ...... $\frac{d}{dx}[x^p]=px^{-1}$
\choice Product Rule ...... $\frac{d}{dx}[f(x)g(x)]=f(x)g'(x)+g(x)f'(x)$
\choice Exponential Rule ...... $\frac{d}{dx}[b^x]=b^x\ln b$
\end{checkboxes}

\vfill

\question[4]
Since $\sin(x)$ has the MacLaurin Series
$\sum_{n=0}^\infty \frac{(-1)^nx^{2n+1}}{(2n+1)!}$,
which of these is the best approximating polynomial for the value of
$\sin(x)$ when $x$ is close to $0$?

\begin{checkboxes}
\choice $1-\frac{x^2}{2}+\frac{x^4}{24}-\frac{x^6}{720}$
\choice $1+x^2+x^3+x^4$
\choice $x-\frac{x^3}{6}+\frac{x^5}{120}-\frac{x^7}{5040}$
\choice $1+x+\frac{x^2}{2}+\frac{x^3}{6}$
\end{checkboxes}

\vfill

\end{questions}

\newpage

\begin{center}
  \textbf{Full Solutions (90 points total)}
\end{center}
\noindent
Please show all work and draw a \framebox{box} around your final answer,
if appropriate. Solutions will be graded according to the rubrics given in
the practice midterm.

\begin{questions}

\setcounter{question}{0}

\question[10]
Find a general formula for the sequence
$
\ds \left\{
\frac{3}{2},
-\frac{4}{4},
\frac{5}{8},
-\frac{6}{16},
\frac{7}{32},
\ldots \right\}
$.

\vfill

\newpage

\question[10]
Does the series $\ds \sum_{n=1}^\infty \frac{(-2)^{n-1}}{3^n}$ converge or
diverge? If it converges, give its sum.

\vfill

\newpage

\question[10]
Determine whether or not
$\ds \sum_{n=1}^{\infty} \frac{\left(-1\right)^{n-1}}{\sqrt{n}}$
is absolutely convergent, conditionally convergent, or divergent.

\vfill

\newpage

\question[10]
Determine whether the series
$\ds \sum_{n = 0}^{\infty} \frac{3n^2}{n^3+1}$
converges or diverges.

\vfill

\newpage

\question[10]
Determine whether the series
$\ds \sum_{n = 2}^{\infty} \frac{\sqrt{n-1}}{2+n^2}$
converges or diverges.

\vfill

\newpage

\question[10]
For what values of $x$ is the series
$\ds \sum_{n=0}^{\infty} \frac{\left(1-x\right)^n}{n+1}$ convergent?
What is its radius of convergence?

\vfill

\newpage

\question[10]
Give a power series representing the function $f(x)=\frac{1}{1+3x}$ and
its radius of convergence.

\vfill

\newpage

\question[10]
Find the Maclaurin series representing the ``hyperbolic cosine'' function
\[f(x)=\cosh(x)=\frac{e^x+e^{-x}}{2}\]

\vfill

\newpage

\question[10]
Evaluate $\ds\int 4xe^x\,dx$.

\vfill

\newpage

\question[5] (BONUS - no partial credit)

A common mistake I see Calculus I students do when taking derivatives is
the following:
\[
  \frac{d}{dx}\left[x^2\sin(x)\right]
    \not=
  \frac{d}{dx}\left[x^2\right]\frac{d}{dx}\left[\sin(x)\right]
    =
  2x\cos(x)
\]
instead of using the product rule to get the correct answer
$x^2\cos(x)+2x\sin(x)$.

Prove that this ``freshman product rule''
\[
  \frac{d}{dx}\left[f(x)g(x)\right]
    =
  f'(x)g'(x)
\]
actually works if $\ds g(x)=e^{\int\frac{f'(x)}{f'(x)-f(x)}\,dx}$.
(An example is when $f(x)=g(x)=e^{2x}$.)




\end{questions}


\end{document}