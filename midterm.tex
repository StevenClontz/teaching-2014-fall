\documentclass[12pt]{exam}

\newcommand{\ds}{\ensuremath{\displaystyle}}

% \printanswers

\begin{document}

\begin{center}
\fbox{\fbox{\parbox{5.5in}{\centering
Calculus II - Fall 2014 - Mr. Clontz - Midterm Exam
}}}
\end{center}
\vspace{0.1in}
\makebox[\textwidth]{Name:\enspace\hrulefill\space 9am / 10am}

\vspace{12pt}

\begin{itemize}
  \item If you completed the practice midterm, turn it in before beginning
        this exam.
  \item This exam is closed-note and closed-book.
  \item The withdrawal deadline is the evening of Tuesday, October 7.
        If you need me to post your grade to Canvas before the deadline,
        please mark this circle:\\
        $\bigcirc$ POST GRADE BEFORE WITHDRAWAL DEADLINE
\end{itemize}

\noindent
Good luck!

\newpage

\begin{center}
  \textbf{Multiple Choice (10 points total)}
\end{center}
\noindent
Please only mark the correct choice for each question.

\begin{questions}

\setcounter{question}{0}
\question[3]
foo

\vfill

\question[3]
foo

\vfill

\question[4]
foo

\vfill

\end{questions}

\newpage

\begin{center}
  \textbf{Full Solutions (90 points total)}
\end{center}
\noindent
Please show all work and draw a \framebox{box} around your final answer,
if appropriate. Solutions will be graded according to the rubrics given in
the practice midterm.

\begin{questions}

\setcounter{question}{0}

\question[10]
Find a general formula for the sequence
$
\ds \left\{
\frac{3}{2},
-\frac{4}{4},
\frac{5}{8},
-\frac{6}{16},
\frac{7}{32},
\ldots \right\}
$.

\vfill

\newpage

\question[10]
Does the series $\ds \sum_{n=1}^\infty \frac{(-2)^{n-1}}{3^n}$ converge or
diverge? If it converges, give its sum.

\vfill

\newpage

\question[10]
Determine whether or not
$\ds \sum_{n=1}^{\infty} \frac{\left(-1\right)^{n-1}}{\sqrt{n}}$
is absolutely convergent, conditionally convergent, or divergent.

\vfill

\newpage

\question[10]\label{convergenceRoundUp}
Determine whether the series
$\ds \sum_{n = 2}^{\infty} \frac{\sqrt{n^5}}{n^3-3}$
converges or diverges.

\textit{Compare with Ch 11 Problems 94-99 and similar problems from earlier
sections.}

\begin{center}
  \begin{tabular}{|c|c|}
    \hline
    Use an identifiable series convergence test. & 2 points \\
    \hline
    Use an appropriate series convergence test. & 2 points \\
    \hline
    Correctly use the chosen series convergence test. & 4 points \\
    \hline
    Identify series as convergent or divergent. & 2 points \\
    \hline
  \end{tabular}
\end{center}

\vfill

\newpage

\question[10]
Determine whether the series
$\ds \sum_{n = 0}^{\infty} \frac{e^n}{(n+1)!}$
converges or diverges.

\textit{See question \#\ref{convergenceRoundUp} for details.}

\vfill

\newpage

\question[10]
Determine whether the series
$\ds \sum_{n = 0}^{\infty} \frac{3+x^2}{x^2(x^2+1)}$
converges or diverges.

\textit{See question \#\ref{convergenceRoundUp} for details.}

\vfill

\newpage

\question[10]
For what values of $x$ is the series
$\ds \sum_{n=1}^{\infty} \frac{\left(2x+1\right)^n}{n^2}$ convergent?
What is its radius of convergence?

\textit{Compare with Ch 11 Problems 103-107.}

\begin{center}
  \begin{tabular}{|c|c|}
    \hline
    Use either the Ratio or Root Test as appropriate. & 2 points \\
    \hline
    Find a correct inequality for convergent $x$-values,
    ignoring endpoints. & 2 points \\
    \hline
    Correctly identify each endpoint as convergent/divergent. & 2 points each \\
    \hline
    Give the correct radius of convergence. & 2 points \\
    \hline
  \end{tabular}
\end{center}

\vfill

\newpage

\question[10]
Give a power series representing the function $f(x)=\frac{2}{2-x}$ and
its radius of convergence.

\textit{Compare with Ch 11 Problems 109-112.}

\begin{center}
  \begin{tabular}{|c|c|}
    \hline
    Set up function in the form $\frac{a}{1-r}$. & 4 points \\
    \hline
    Set up the geometric series $\sum_{n=0}^\infty ar^n$. & 4 points \\
    \hline
    Give the radius of convergence. & 2 points \\
    \hline
  \end{tabular}
\end{center}

\vfill

\newpage

\question[10]
The function $f(x)=\frac{3x}{1-x}$ is represented by the power series
$\sum_{n=0}^\infty 3x^{n+1}$. Give a power series representing the function
$f'(x)=\frac{3}{(1-x)^2}$.

\textit{Compare with Ch 11 Problems 114-116.}

\begin{center}
  \begin{tabular}{|c|c|}
    \hline
    Attempt to differentiate/integrate the given series as appropriate.
    & 4 points \\
    \hline
    Correctly differentiate/integrate the given series as appropriate.
    & 4 points \\
    \hline
    Give correctly formatted series for final answer. & 2 points \\
    \hline
  \end{tabular}
\end{center}

\vfill

\newpage

\question[10]
Find the Maclaurin series representing the function $f(x)=e^{2x}$.

\textit{Compare with Ch 11 Problems 119-122.}

\begin{center}
  \begin{tabular}{|c|c|}
    \hline
    Use MacLaurin series formula. & 2 points \\
    \hline
    Compute derivatives $f^{(n)}(x)$. & 2 points \\
    \hline
    Find formula for $f^{(n)}(0)$ (possibly splitting up odds/evens).
    & 4 points \\
    \hline
    Give correctly formatted series for final answer. & 2 points \\
    \hline
  \end{tabular}
\end{center}

\vfill

\newpage

\question[10]
Evaluate $\ds\int 3x^2\cos(x)\,dx$.

\textit{Compare with Ch 7 Problems 3-7.}

\begin{center}
  \begin{tabular}{|c|c|}
    \hline
    Set up correct $u$ and $dv$. & 2 points \\
    \hline
    Compute correct $du$ and $v$. & 2 points \\
    \hline
    Apply integration by parts to get solvable $uv-\int v\,du$. & 4 points \\
    \hline
    Find correct final answer (possibly using int. by parts
    multiple times). & 2 points \\
    \hline
  \end{tabular}
\end{center}

\vfill

\newpage

\question[10]
Evaluate $\ds\int \tan^7(y)\sec^4(y)\,dy$.

\textit{Compare with Ch 7 Problems 9,10,14,15}

\begin{center}
  \begin{tabular}{|c|c|}
    \hline
    Use correct trigonometric identities. & 3 points \\
    \hline
    Rewrite integral with single trig function and its derivative. & 3 points \\
    \hline
    Use $u$ substitution to eliminate trig functions. & 2 points \\
    \hline
    Find correct final answer. & 2 points \\
    \hline
  \end{tabular}
\end{center}

\vfill





\end{questions}


\end{document}