\documentclass[12pt]{exam}

\newcommand{\ds}{\ensuremath{\displaystyle}}

% \printanswers

\begin{document}

\begin{center}
\fbox{\fbox{\parbox{5.5in}{\centering
Calculus II - Fall 2014 - Mr. Clontz - PRACTICE Midterm
}}}
\end{center}
\vspace{0.1in}
\makebox[\textwidth]{Name:\enspace\hrulefill\space 9am / 10am}

\vspace{12pt}

\noindent About your midterm and this practice midterm:
\begin{itemize}
  \item
    The first few questions will be multiple-choice questions
    covering basic definitions, theorems, and other concepts in sections
    11.1-11.10, 7.1, and 7.2.
    These will total 10 points of the midterm, and are not covered
    by the practice midterm.
  \item
    The other 90 points of the midterm are based on 9 of the questions
    asked on this practice midterm. These questions will require full solutions
    and will be given partial credit based on the rubric for each question
    in this practice midterm. The rubric will not be given on the actual
    midterm.
  \item
    Minor errors in a solution not specifically covered by the rubric will have
    between 0 and 2 points deducted. Some problems may have solutions not
    outlined by the rubric - these will be graded as fairly as possible.
  \item
    A review for the midterm will be held on Friday, October 3 during lecture.
    Students will receive presentation credit for solving problems from the
    practice midterm.
  \item
    The midterm will take place during lecture on Monday, October 6. Up to
    ten bonus points will be awarded for turning in a printed copy of
    this practice midterm with full solutions before taking the midterm.
\end{itemize}

\newpage

\begin{questions}

\setcounter{question}{0}
\question[10]
Write the first five terms of the following sequence:
$\ds\left\{\frac{3n}{2^n}\right\}_{n=0}^\infty$

\textit{Compare with Ch 11 Problem 16.}

\begin{center}
  \begin{tabular}{|c|c|}
    \hline
    Write a correct term of the sequence. & 2 points each \\
    \hline
  \end{tabular}
\end{center}

\vfill

\newpage

\question[10]
Find a general formula for the sequence
$
\ds \left\{
\frac{2}{3},
\frac{-4}{5},
\frac{8}{7},
\frac{-16}{9},
\frac{32}{11},
\ldots \right\}
$.

\textit{Compare with Ch 11 Problem 17.}

\begin{center}
  \begin{tabular}{|c|c|}
    \hline
    Formula yields correct denominators. & 4 points \\
    \hline
    Formula yields correct numerators. & 4 points \\
    \hline
    Formula yields correct signs (positive/negative). & 2 points \\
    \hline
  \end{tabular}
\end{center}

\vfill

\newpage

\question[10]
Determine whether the sequence $\ds \left( \frac{1-n}{3n+7} \right)$
is convergent or divergent.  If it is convergent, what does it converge to?

\textit{Compare with Ch 11 Problems 31-38.}

\begin{center}
  \begin{tabular}{|c|c|}
    \hline
    Correctly identify as convergent/divergent. & 2 points \\
    \hline
    If convergent, found correct value.
    If divergent, identifed as divergent. & 2 points \\
    \hline
    Use correct techniques (factoring/L'Hopital/etc.) to compute
    limit. & 6 points \\
    \hline
  \end{tabular}
\end{center}

\vfill

\newpage

\question[10]
Does the series $\ds \sum_{n=1}^\infty \frac{4^n}{5^{n-1}}$ converge or
diverge? If it converges, give its sum.

\textit{Compare with Ch 11 Problems 53,54.}

\begin{center}
  \begin{tabular}{|c|c|}
    \hline
    Identify as a geometric series. & 2 points \\
    \hline
    Identify $a$ and $r$. & 2 points \\
    \hline
    Check if $|r|<1$ or $|r|\geq 1$. & 2 points \\
    \hline
    If $|r|\geq 1$, identify as divergent.
    If $|r|<1$, use formula $\frac{a}{1-r}$ to compute sum. & 4 points \\
    \hline
  \end{tabular}
\end{center}

\vfill

\newpage

\question[10]
Determine whether or not
$\ds \sum_{n=1}^{\infty} \frac{\left(-1\right)^{n-1}}{n^{3/2}}$
is absolutely convergent, conditionally convergent, or divergent.

\textit{Compare with Ch 11 Problems 83,84.}

\begin{center}
  \begin{tabular}{|c|c|}
    \hline
    Check the series of absolute values. & 2 points \\
    \hline
    Identify series of absolute values as convergent/divergent. & 2 points \\
    \hline
    If necessary, check original series and identify as
    convergent/divergent. & 2 points \\
    \hline
    Identify series as absolutely convergent, conditionally convergent,
    or divergent. & 4 points \\
    \hline
  \end{tabular}
\end{center}

\vfill

\newpage

\question[10]\label{convergenceRoundUp}
Determine whether the series
$\ds \sum_{n = 2}^{\infty} \frac{\sqrt{n^5}}{n^3-3}$
converges or diverges.

\textit{Compare with Ch 11 Problems 94-99 and similar problems from earlier
sections.}

\begin{center}
  \begin{tabular}{|c|c|}
    \hline
    Use an identifiable series convergence test. & 2 points \\
    \hline
    Use an appropriate series convergence test. & 2 points \\
    \hline
    Correctly use the chosen series convergence test. & 4 points \\
    \hline
    Identify series as convergent or divergent. & 2 points \\
    \hline
  \end{tabular}
\end{center}

\vfill

\newpage

\question[10]
Determine whether the series
$\ds \sum_{n = 0}^{\infty} \frac{e^n}{(n+1)!}$
converges or diverges.

\textit{See question \#\ref{convergenceRoundUp} for details.}

\vfill

\newpage

\question[10]
Determine whether the series
$\ds \sum_{n = 0}^{\infty} \frac{3+x^2}{x^2(x^2+1)}$
converges or diverges.

\textit{See question \#\ref{convergenceRoundUp} for details.}

\vfill

\newpage

\question[10]
For what values of $x$ is the series
$\ds \sum_{n=1}^{\infty} \frac{\left(2x+1\right)^n}{n^2}$ convergent?
What is its radius of convergence?

\textit{Compare with Ch 11 Problems 103-107.}

\begin{center}
  \begin{tabular}{|c|c|}
    \hline
    Use either the Ratio or Root Test as appropriate. & 2 points \\
    \hline
    Find a correct inequality for convergent $x$-values,
    ignoring endpoints. & 2 points \\
    \hline
    Correctly identify each endpoint as convergent/divergent. & 2 points each \\
    \hline
    Give the correct radius of convergence. & 2 points \\
    \hline
  \end{tabular}
\end{center}

\vfill

\newpage

\question[10]
Give a power series representing the function $f(x)=\frac{2}{2-x}$ and
its radius of convergence.

\textit{Compare with Ch 11 Problems 109-112.}

\begin{center}
  \begin{tabular}{|c|c|}
    \hline
    Set up function in the form $\frac{a}{1-r}$. & 4 points \\
    \hline
    Set up the geometric series $\sum_{n=0}^\infty ar^n$. & 4 points \\
    \hline
    Give the radius of convergence. & 2 points \\
    \hline
  \end{tabular}
\end{center}

\vfill

\newpage

\question[10]
The function $f(x)=\frac{3x}{1-x}$ is represented by the power series
$\sum_{n=0}^\infty 3x^{n+1}$. Give a power series representing the function
$f'(x)=\frac{3}{(1-x)^2}$.

\textit{Compare with Ch 11 Problems 114-116.}

\begin{center}
  \begin{tabular}{|c|c|}
    \hline
    Attempt to differentiate/integrate the given series as appropriate.
    & 4 points \\
    \hline
    Correctly differentiate/integrate the given series as appropriate.
    & 4 points \\
    \hline
    Give correctly formatted series for final answer. & 2 points \\
    \hline
  \end{tabular}
\end{center}

\vfill

\newpage

\question[10]
Find the Maclaurin series representing the function $f(x)=e^{2x}$.

\textit{Compare with Ch 11 Problems 119-122.}

\begin{center}
  \begin{tabular}{|c|c|}
    \hline
    Use MacLaurin series formula. & 2 points \\
    \hline
    Compute derivatives $f^{(n)}(x)$. & 2 points \\
    \hline
    Find formula for $f^{(n)}(0)$ (possibly splitting up odds/evens).
    & 4 points \\
    \hline
    Give correctly formatted series for final answer. & 2 points \\
    \hline
  \end{tabular}
\end{center}

\vfill

\newpage

\question[10]
Evaluate $\ds\int 3x^2\cos(x)\,dx$.

\textit{Compare with Ch 7 Problems 3-7.}

\begin{center}
  \begin{tabular}{|c|c|}
    \hline
    Set up correct $u$ and $dv$. & 2 points \\
    \hline
    Compute correct $du$ and $v$. & 2 points \\
    \hline
    Apply integration by parts to get solvable $uv-\int v\,du$. & 4 points \\
    \hline
    Find correct final answer (possibly using int. by parts
    multiple times). & 2 points \\
    \hline
  \end{tabular}
\end{center}

\vfill

\newpage

\question[10]
Evaluate $\ds\int \tan^7(y)\sec^4(y)\,dy$.

\textit{Compare with Ch 7 Problems 9,10,14,15}

\begin{center}
  \begin{tabular}{|c|c|}
    \hline
    Use correct trigonometric identities. & 3 points \\
    \hline
    Rewrite integral with single trig function and its derivative. & 3 points \\
    \hline
    Use $u$ substitution to eliminate trig functions. & 2 points \\
    \hline
    Find correct final answer. & 2 points \\
    \hline
  \end{tabular}
\end{center}

\vfill





\end{questions}


\end{document}